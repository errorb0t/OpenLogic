% Part: first-order-logic
% Chapter: introduction
% Section: satisfaction

\documentclass[../../../include/open-logic-section]{subfiles}

\begin{document}

\olfileid{fol}{int}{sat}

\olsection{Satisfaction}

We can already skip ahead to the semantics of first-order logic once
we know what !!{formula}s are: here, the basic definition is that of 
!!a{structure}. For our simple language, !!a{structure}~$\Struct M$ has
just three components: a non-empty set $\Domain{M}$ called the
\emph{!!{domain}}, what $\Obj a$ picks out in~$\Struct M$, and what
$\Obj P$ is true of in~$\Struct M$.  The object picked out by~$\Obj a$
is denoted~$\Assign{\Obj a}{M}$ and the set of things $\Obj P$ is true
of by~$\Assign{\Obj P}{M}$. !!^a{structure}~$\Struct{M}$ consists of
just these three things: $\Domain{M}$, $\Assign{\Obj a}{M} \in
\Domain{M}$ and $\Assign{\Obj P}{M} \subseteq \Domain{M}$. The general
case will be more complicated, since there will be many !!{predicate}s
and !!{constant}s, the !!{constant}s can have more than one place, and
there will also be !!{function}s.

This is enough to give a definition of satisfaction for !!{formula}s
that don't contain !!{variable}s.  The idea is to give an inductive
definition that mirrors the way we have defined !!{formula}s. We
specify when an atomic formula is satisfied in~$\Struct{M}$, and then
when, e.g., $\lnot !A$ is satisfied in~$\Struct{M}$ on the basis of
whether or not $!A$ is satisfied in~$\Struct{M}$. E.g., we could
define:
\begin{enumerate}
  \item $\Atom{\Obj P}{\Obj a}$ is satisfied in~$\Struct{M}$ iff
  $\Assign{\Obj a}{M} \in \Assign{\Obj P}{M}$.
  \item $\lnot !A$ is satisfied in~$\Struct{M}$ iff $!A$ is not
  satisfied in~$\Struct{M}$.
  \item $(!A \land !B)$ is satisfied in~$\Struct{M}$ iff $!A$ is
  satisfied in~$\Struct{M}$, and $!B$ is satisfied in~$\Struct{M}$ as
  well.
\end{enumerate}
Let's say that $\Domain{M} = \{0, 1, 2\}$, $\Assign{\Obj a}{M} = 1$,
and $\Assign{\Obj P}{M} = \{1, 2\}$. This definition would tell us
that $\Atom{\Obj P}{\Obj a}$ is satisfied in~$\Struct{M}$ (since
$\Assign{\Obj a}{M} =  1 \in \{1,2\} = \Assign{\Obj P}{M}$). It tells
us further that $\lnot \Atom{\Obj P}{\Obj a}$ is not satisfied
in~$\Struct{M}$, and that in turn $\lnot\lnot \Atom{\Obj P}{\Obj
a}$ is and $(\lnot \Atom{\Obj P}{\Obj a} \land \Atom{\Obj P}{\Obj a})$
is not satisfied, and so on.

The trouble comes when we want to give a definition for the
quantifiers: we'd like to say something like, ``$\lexists[\Obj
v_0][\Atom{\Obj P}{\Obj v_0}]$ is satisfied iff $\Atom{\Obj P}{\Obj
v_0}$ is satisfied.'' But the !!{structure}~$\Struct{M}$ doesn't tell
us what to do about !!{variable}s. What we actually want to say is
that $\Atom{\Obj P}{\Obj v_0}$ is satisfied \emph{for some value
of~$\Obj v_0$}.  To make this precise we need a way to assign
!!{element}s of~$\Domain{M}$ not just to $\Obj a$ but also to~$\Obj
v_0$. To this end, we introduce !!{variable} \emph{assignments}.
!!^a{variable} assignment is simply a function~$s$ that maps
!!{variable}s to !!{element}s of~$\Domain{M}$ (in our example, to one
of $1$, $2$, or~$3$).  Since we don't know beforehand which
!!{variable}s might appear in !!a{formula} we can't limit which
!!{variable}s $s$ assigns values to. The simple solution is to
require that $s$ assigns values to \emph{all} !!{variable}s~$\Obj
v_0$, $\Obj v_1$, \dots\@ We'll just use only the ones we need.

Instead of defining satisfaction of !!{formula}s just relative to
!!a{structure}, we'll define it relative to
!!a{structure}~$\Struct{M}$ \emph{and} !!a{variable} assignment~$s$,
and write $\Sat{M}{!A}[s]$ for short. Our definition will now include
an additional clause to deal with atomic !!{formula}s containing
!!{variable}s:
\begin{enumerate}
  \item $\Sat{M}{\Atom{\Obj P}{\Obj a}}[s]$ iff
  $\Assign{\Obj a}{M} \in \Assign{\Obj P}{M}$.
  \item $\Sat{M}{\Atom{\Obj P}{\Obj v_i}}[s]$ iff
  $s(\Obj v_i) \in \Assign{\Obj P}{M}$.
  \item $\Sat{M}{\lnot !A}[s]$ iff not $\Sat{M}{!A}[s]$.
  \item $\Sat{M}{(!A \land !B)}[s]$ iff $\Sat{M}{!A}[s]$ and $\Sat{M}{!B}[s]$.
\end{enumerate}
Ok, this solves one problem: we can now say when $\Struct{M}$
satisfies $\Atom{\Obj P}{\Obj v_0}$ for the value~$s(\Obj v_0)$. To
get the definition right for $\lexists[\Obj v_0][\Atom{\Obj P}{\Obj
v_0}]$ we have to do one more thing:  We want to have that
$\Sat{M}{\lexists[\Obj v_0][\Atom{\Obj P}{\Obj v_0}]}[s]$ iff
$\Sat{M}{\Atom{\Obj P}{\Obj v_0}}[s']$ for \emph{some} way $s'$ of
assigning a value to~$\Obj v_0$. But the value assigned to~$\Obj v_0$
does not necessarily have to be the value that $s(\Obj v_0)$ picks
out.  We'll introduce a notation for that: if $m \in \Domain{M}$, then
we let $\Subst{s}{m}{\Obj v_0}$ be the assignment that is just
like~$s$ (for all !!{variable}s other than~$\Obj v_0$), except to
$\Obj v_0$ it assigns~$m$. Now our definition can be:
\begin{enumerate}\setcounter{enumi}{4}
  \item $\Sat{M}{\lexists[\Obj v_i][!A]}[s]$ iff
  $\Sat{M}{!A}[\Subst{s}{m}{\Obj v_i}]$ for some~$m \in \Domain{M}$.
\end{enumerate}
Does it work out? Let's say we let $s(\Obj v_i) = 0$ for all~$i \in
\Nat$. $\Sat{M}{\lexists[\Obj v_0][\Atom{\Obj P}{\Obj v_0}]}[s]$ iff
there is an $m \in \Domain{M}$ so that $\Sat{M}{\Atom{\Obj P}{\Obj
v_0}}[\Subst{s}{m}{\Obj v_0}]$. And there is: we can choose $m = 1$ or
$m = 2$. Note that this is true even if the value~$s(\Obj v_0)$
assigned to~$\Obj v_0$ by $s$ itself---in this case, $0$---doesn't do
the job. We have $\Sat{M}{\Atom{\Obj P}{\Obj v_0}}[\Subst{s}{1}{\Obj
v_0}]$ but not $\Sat{M}{\Atom{\Obj P}{\Obj v_0}}[s]$.

If this looks confusing and cumbersome: it is. But the added
complexity is required to give a precise, inductive definition of
satisfaction for all !!{formula}s, and we need something like it to
precisely define the semantic notions.  There are other ways of doing
it, but they are all equally (in)elegant.

\end{document}
