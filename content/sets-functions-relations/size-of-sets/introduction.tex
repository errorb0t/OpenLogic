% Part:sets-functions-relations
% Chapter: size-of-sets
% Section: introduction

\documentclass[../../../include/open-logic-section]{subfiles}

\begin{document}

\olfileid{sfr}{siz}{int}

\olsection{Einführung}

Als Georg Cantor in den 1870er-Jahren seine Mengenlehre entwickelte, war eines seiner Ziele, die Idee einer unendlichen Sammlung salonfähig zu machen - eine
aktuale Unendlichkeit, wie man im Mittelalter gesagt hätte.  Ein wichtiger Teil davon war seine
Behandlung der \emph{Größenordnungen} der verschiedenen Mengen. Wenn $a$, $b$ und $c$
alle verschieden sind, dann ist die Menge $\{a, b, c\}$ intuitiv \emph{größer}
als $\{a, b\}$. Aber was ist mit unendlichen Mengen? Sind sie alle gleich groß? Es stellt sich heraus, dass sie es nicht sind.

Die erste wichtige Idee hier ist die einer Aufzählung. Wir können
jede endliche Menge auflisten, indem wir alle ihre Elemente auflisten. Für einige
unendlichen Mengen können wir auch alle ihre Elemente auflisten, wenn wir zulassen, dass die
Liste selbst unendlich sein kann. Solche Mengen nennt man abzählbar.
Das überraschende Ergebnis von Cantor, das wir am Ende dieses Kapitels vollständig verstehen werden, war, dass einige unendliche Mengen nicht abzählbar sind.

\end{document}