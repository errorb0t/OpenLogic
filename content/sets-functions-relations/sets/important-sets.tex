% Teil: sets-functions-relations
% Kapitel: sets
% Abschnitt: important-sets

\documentclass[../../../include/open-logic-section]{subfiles}

\begin{document}

\olfileid{sfr}{set}{imp}
\olsection{Einige wichtige Mengen}

\begin{ex}
Wir werden uns hauptsächlich mit Mengen beschäftigen, deren Elemente
mathematische Objekte sind. Manche dieser Mengen sind wichtig genug,
dass wir ihnen Namen geben. Hier sind vier davon:
\begin{multline*}
    \Nat = \{0, 1, 2, 3, \ldots\} \\
    \shoveright{\text{die Menge der natürlichen Zahlen}} \\
    \shoveleft{\Int = \{\ldots, -2, -1, 0, 1, 2, \ldots\}} \\
    \shoveright{\text{die Menge der ganzen Zahlen}} \\
    \shoveleft{\Rat = \Setabs{\frac{m}{n}}{m, n \in \Int\text{ und }n \neq 0}} \\
    \shoveright{\text{die Menge der rationalen Zahlen}} \\
    \shoveleft{\Real = (-\infty, \infty)} \\
    \text{die Menge der reellen Zahlen (das Kontinuum)} \\
\end{multline*}
Dies sind alles \emph{unendliche} Mengen, das heißt, sie haben jeweils
unendlich viele Elemente.

Während wir von einer Menge zur nächsten schreiten, fügen wir unserem 
Bestand \emph{mehr} Zahlen hinzu.
Es sollte in der Tat klar sein, dass $\Nat \subseteq \Int
\subseteq \Rat \subseteq \Real$: Schließlich ist jede natürliche Zahl eine
ganze Zahl; jede ganze Zahl ist eine rationale Zahl; und jede rationale Zahl ist
eine reelle Zahl.
Ebenso sollte es klar sein, dass $\Nat \subsetneq \Int \subsetneq
\Rat$, da $-1$ eine ganze Zahl, aber keine natürliche Zahl ist, und
$\frac{1}{2}$ ist rational, aber nicht ganzzahlig. Es ist weniger offensichtlich
dass $\Rat \subsetneq \Real$, d.h. dass es einige reelle Zahlen gibt,
die nicht rational sind\oliflabeldef{sfr:arith:real:realline}{, aber
Wir werden darauf in \olref[arith][real]{realline}}{} zurückkommen. 

In der Mathematik herrscht keine Einigkeit
darüber, ob die natürlichen Zahlen $0$ einschließen. Wir zählen $0$ dazu.
Wir werden manchmal aber auch die Menge der positiven ganzen Zahlen $\PosInt = \{1,
2, 3, \dots\}$ verwenden. Außerdem bezeichnen wir die Menge der ersten beiden natürlichen Zahlen
Zahlen $\Bin = \{0, 1\}$.
\end{ex}

\begin{tagblock}{compsci}
\begin{ex}[Strings] 
Ein weiteres interessantes Beispiel ist die Menge $A^{*}$ von \emph{endlichen Zeichenfolgen}
über einem Alphabet $A$: jede endliche Folge von Elementen von~$A$
ist eine Zeichenfolge über $A$. Wir zählen die \emph{leere Zeichenfolge $\Lambda$}
zu den Zeichenfolgen über~$A$, für jedes Alphabet~$A$. Zum Beispiel:
\begin{multline*}
\Bin^*
=\{\Lambda,0,1,00,01,10,11,\\
000,001,010,011,100,101,110,111,0000,\ldots\}.
\end{multline*}
Wenn $x=x_{1}\ldots x_{n}\in A^{*}$ eine Zeichenfolge ist, die aus $n$
``Buchstaben'' aus $A$ besteht, dann sagen wir: Die \emph{Länge} der Zeichenfolge ist~$n$
und schreiben $\len{x}=n$ (für \emph{length}).
\end{ex}
\end{tagblock}

\begin{ex}[Unendliche Folgen]
Für jede Menge $A$ können wir auch die Menge~$A^\omega$ der unendlichen
Folgen von Elementen aus~$A$ betrachten. Eine unendliche Folge
$a_1a_2a_3a_4\dots$ besteht aus einer einseitig unendlichen Liste von Objekten,
von denen jedes ein Element aus~$A$ ist.
\end{ex}

\end{document}
