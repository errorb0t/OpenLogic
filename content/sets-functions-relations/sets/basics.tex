% Teil: sets-functions-relations
% Kapitel: Mengen
% Abschnitt: Grundlagen

\documentclass[../../../include/open-logic-section]{subfiles}

\begin{document}

\olfileid{sfr}{set}{bas}
\olsection{Extensionalität}

Eine \emph{Menge} ist eine Sammlung von Objekten, die als ein einziges
Objekt behandelt wird. Die Objekte, aus denen die Menge besteht, werden \emph{Elemente} oder
\emph{Mitglieder} der Menge genannt. Wenn $x$ !!a{Element} einer Menge~$a$ ist,
schreiben wir $x \in a$; wenn nicht, schreiben wir $x \notin a$. Die Menge, die keine
!!{Element}e hat, wird die \emph{leere} Menge genannt und
bezeichnet mit~``$\emptyset$''.

\begin{explain}
Es spielt keine Rolle, wie wir die Menge \emph{angeben}, oder wie wir
ihre !!{Element}e anordnen, oder wie oft wir ihre !!{Element}e zählen. Alles, was zählt, ist, was die !!{Element}e
sind. Das bringen wir mit dem folgenden Prinzip zum Ausdruck.
\end{explain}

\begin{defn}[Extensionalität]
  Wenn $A$ und $B$ Mengen sind, dann ist $A = B$ genau dann, wenn
  jedes !!{Element} von~$A$ auch !!ein{Element} von~$B$ ist, und umgekehrt.
\end{defn}

Die Extensionalität erlaubt eine gewisse Notation. Im Allgemeinen, wenn wir einige
Objekte $a_{1}$, \dots, $a_{n}$ haben, dann ist $\{a_{1}, \dots, a_{n}\}$
\emph{die} Menge, deren !!{Element}e $a_1, \dots, a_n$ sind. Wir betonen das Wort ``emph{die}'', da es aufgrund der Extensionalität
nur \emph{eine} solche Menge geben kann. In der Tat erlaubt die Extensionalität auch das
Folgende:
  \[
    \{a, a, b\} = \{a, b\} = \{b,a\}.
  \] 
Das zeigt, dass wir uns bei der Betrachtung von Mengen nicht um die Reihenfolge ihrer Elemente kümmern müssen, oder darum, wie oft sie angegeben sind.

\begin{tagblock}{novice}
\begin{ex}
Wann immer wir eine Ansammlung von Objekten haben, können wir sie in einer Menge zusammenfassen.
 Die Menge von Richards Geschwistern zum Beispiel ist eine Menge, die
eine Person enthält, und wir können sie als $S=\{\textrm{Ruth}\}$ schreiben.
Die Menge der positiven ganzen Zahlen kleiner als $4$ ist $\{1, 2, 3\}$, aber sie
kann auch als $\{3, 2, 1\}$ oder sogar als $\{1, 2, 1, 2, 3\}$ geschrieben werden.
Das sind alle die gleiche Menge, wegen der Extensionalität. Denn jedes !!{Element}
von $\{1, 2, 3\}$ ist auch ein !!{Element} von $\{3, 2, 1\}$ (und von $\{1,
2, 1, 2, 3\}$) und umgekehrt.
\end{ex} 
\end{tagblock}

Häufig werden wir eine Menge durch eine Eigenschaft spezifizieren, die ihre !!{Element}e
gemeinsam haben. Wir verwenden dafür die folgende Kurzschreibweise:
$\Setabs{x}{\phi(x)}$, wobei das $\phi(x)$ für die Eigenschaft steht,
die~$x$ haben muss, um zu den !!{Element}en der Menge zu zählen.

\begin{tagblock}{novice}
\begin{ex}
In unserem Beispiel hätten wir $S$ auch so angeben können:
\[
S = \Setabs{x}{x \text{ist ein Geschwister von Richard}}.
\]
\end{ex}
\end{tagblock}

\begin{tagblock}{math}
\begin{ex}
Eine Zahl wird \emph{vollkommen} genannt, wenn sie gleich der Summe ihrer
echten Teiler (d.h. Zahlen, die sie restlos teilen, aber nicht
identisch mit der Zahl sind). Zum Beispiel ist $6$ vollkommen, weil ihre
echten Teiler $1$, $2$ und~$3$ sind und $6 = 1 + 2 + 3$ gilt. $6$ ist die einzige positive ganze Zahl kleiner als $10$, die vollkommen ist. Also
können wir mit Hilfe der Extensionalität sagen:
\[
	\{6\} = \Setabs{x}{x\text{ ist vollkommen und }0 \leq x \leq 10}
\]
Wir lesen die Notation auf der rechten Seite als ``die Menge der $x$'s, so dass $x$
vollkommen ist und $0 \leq x \leq 10$''. Die Identität hier bekräftigt, diese unterschiedlich 
geschriebenen Mengen letztlich nur von ihren Elementen abhängen.
Allgemeiner ausgedrückt, garantiert die Extensionalität, dass es immer
nur eine Menge von $x$ gibt, für die $\phi(x)$ gilt.
Die Extensionalität rechtfertigt also die Bezeichnung von 
$\Setabs{x}{\phi(x)}$ als \emph{die} Menge aller $x$ sodass~$\phi(x)$.
\end{ex}
\end{tagblock}

Die Extensionalität gibt uns eine Möglichkeit zu beweisen, dass Mengen identisch sind: Um zu zeigen, dass $A = B$ gilt, muss man zeigen, dass, wann immer $x \in A$ ist, auch $x \in B$ ist,
und wenn $y \in B$ ist, dann auch $y \in A$ gilt.

\begin{prob}
Beweisen Sie, dass es höchstens eine leere Menge gibt, d.h. zeigen Sie, dass wenn $A$ und $B$
Mengen ohne !!{Element}e sind, dann $A = B$ gilt.
\end{prob}

\end{document}