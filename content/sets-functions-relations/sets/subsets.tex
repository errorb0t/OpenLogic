% Teil: sets-functions-relations
% Kapitel: Mengen
% Abschnitt: subsets

\documentclass[../../../include/open-logic-section]{subfiles}

\begin{document}

\olfileid{sfr}{set}{sub}
\olsection{Teilmengen und Potenzmengen}

\begin{explain}
Wir werden Mengen oft vergleichen wollen. Und eine offensichtliche Art des Vergleichs
könnte folgendermaßen lauten: \emph{Alles in der einen Menge ist auch in der anderen}.
Diese Situation ist so wichtig, dass wir für sie
eine neue Notation einzuführen.
\end{explain}

\begin{defn}[Teilmenge]
Wenn jedes !!{Element} einer Menge $A$ auch !!ein{Element} von~$B$ ist, dann sagen wir,
dass $A$ eine \emph{Teilmenge} von~$B$ ist, und schreiben $A \subseteq B$. Wenn
$A$ keine Teilmenge von~$B$ ist, schreiben wir $A \not\subseteq B$.
Wenn $A \subseteq B$ gilt, aber $A \neq B$ ist, schreiben wir $A \subsetneq B$ und sagen,
dass $A$ eine \emph{echte Teilmenge} von $B$ ist.
\end{defn}

\begin{ex}
Jede Menge ist eine Teilmenge ihrer selbst, und $\emptyset$ ist eine Teilmenge jeder
Menge. Die Menge der geraden Zahlen ist eine Teilmenge der Menge der natürlichen
Zahlen. Außerdem: $\{ a, b \} \subseteq \{ a, b, c \}$. Aber $\{ a, b, e
\}$ ist keine Teilmenge von $\{ a, b, c \}$.
\end{ex}

\begin{ex}
Die Zahl $2$ ist ein !!{Element} der Menge der ganzen Zahlen, während die
Menge der geraden Zahlen eine Teilmenge der Menge der ganzen Zahlen ist. Allerdings kann eine 
Menge jedoch sowohl ein !!{Element} als auch eine Teilmenge einer anderen Menge sein, z.B. $\{0\} \in \{0, \{0\}\}$ und auch $\{0\} \subseteq \{0,
\{0\}\}$.
\end{ex}

Die Extensionalität gibt ein Kriterium für die Gleichheit zweier Mengen 
vor: $A = B$ genau dann, wenn
jedes !!{Element} von~$A$ auch !!ein{Element} von~$B$ ist und umgekehrt.
Die Definition von "Teilmenge" legt $A \subseteq B$ genau wie die
erste Hälfte dieses Kriteriums fest: Jedes !!{Element} von~$A$ ist auch
!!ein{Element} von~$B$. Natürlich gilt die Definition auch, wenn wir
$A$ und $B$ vertauschen. Das heißt, $B \subseteq A$, wenn jedes !!{Element}
von~$B$ auch !!ein{Element} von~$A$ ist. Und das wiederum ist genau die zweite Hälfte
der Extensionalität. Mit anderen Worten, Extensionalität
bedeutet, dass Mengen gleich sind genau dann, wenn sie Teilmengen voneinander sind.

\begin{prop}
$A = B$ genau dann, wenn sowohl $A \subseteq B$ als auch $B \subseteq A$ sind.
\end{prop}

Jetzt ist auch eine gute Gelegenheit, ein paar weitere
hilfreiche Notationen einzuführen. Bei der Definition, wann $A$ eine Teilmenge von~$B$ ist,
haben wir gesagt, dass
``jedes !!{Element} von~$A$ ist \dots,'' und haben für die ``$\dots$''
``!!a{Element} von $B$'' eingesetzt. Aber das ist eine so häufige \emph{Form} von
Ausdrucks, dass es hilfreich sein wird, eine formale Notation dafür einzuführen.

\begin{defn}\ollabel{forallxina}
$(\forall x \in A)\phi$ ist eine Abkürzung für $\forall x(x \in A \lif
\phi)$. Analog ist $(\exists x \in A)\phi$ rine Abkürzung für $\exists x(x
\in A \land \phi)$. 
\end{defn}

Mit dieser Notation können wir sagen, dass $A \subseteq B$ ist, wenn $(\forall
x \in A)x \in B$. 

Wir gehen nun dazu über, eine bestimmte Art von Menge zu betrachten: die Menge aller
Teilmengen einer gegebenen Menge. 

\begin{defn}[Potenzmenge]
Die Menge, die aus allen Teilmengen einer Menge~$A$ besteht, nennt man die
\emph{Potenzmenge von}~$A$, geschrieben $\Pow{A}$.
  \[
    \Pow{A} = \Setabs{B}{B \subseteq A} 
  \]
\end{defn}

\begin{ex}
Was sind alle möglichen Teilmengen von $\{ a, b, c \}$? Sie sind:
$\emptyset$, $\{a \}$, $\{b\}$, $\{c\}$, $\{a, b\}$, $\{a, c\}$, $\{b,
c\}$, $\{a, b, c\}$. Die Menge all dieser Teilmengen ist
$\Pow{\{a,b,c\}}$:
\[
\Pow{\{a, b, c \}} = \{\emptyset, \{a \}, \{b\}, \{c\}, \{a, b\},
\{b, c\}, \{a, c\}, \{a, b, c\}}
\]
\end{ex}

\begin{prob}
Listen Sie alle Teilmengen von $\{a, b, c, d\}$ auf.
\end{prob}

\begin{prob}
Zeigen Sie, dass wenn $A$ $n$ !!{Element}e hat, dann hat $\Pow{A}$ $2^n$
!!{Element}e.
\end{prob}

\end{document}
