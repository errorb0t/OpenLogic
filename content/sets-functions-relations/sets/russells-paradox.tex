% Teil: sets-functions-relations
% Kapitel: mengen
% Abschnitt: russells-paradox

\documentclass[../../../include/open-logic-section]{subfiles}


\begin{document}

\olfileid{sfr}{set}{rus}
\olsection{Die Russellsche Antinomie}

Die Extensionalität gestattet die Notation $\Setabs{x}{\phi(x)}$, für
\emph{die} Menge der $x$'s, so dass~$\phi(x)$. Allerdings ist alles, was
Extensionalität \emph{wirklich} gestattet, der folgende Gedanke:
\emph{Wenn} es eine Menge gibt, deren Mitglieder genau die $\phi$s sind,
\emph{dann} gibt es nur eine solche Menge. Anders ausgedrückt: Für ein festes~$\phi$
ist die Menge $\Setabs{x}{\phi(x)}$ eindeutig, \emph{wenn sie
existiert}.

Aber diese Bedingung ist wichtig!{} Entscheidend ist, dass sich nicht jede Eigenschaft
für \emph{Abstraktion} eignet. Das heißt: Einige Eigenschaften definieren \emph{keine} Mengen.
Täten sie das doch, entstünden Widersprüche. Das berühmteste Beispiel hierfür ist die Russellsche
Antinomie.

Mengen können Elemente anderer Mengen sein - zum Beispiel besteht die Potenzmenge
einer Menge~$A$ aus Mengen. Und so macht es Sinn zu fragen oder
zu untersuchen, ob eine Menge Element einer anderen Menge ist. Kann eine Menge
ein Mitglied von sich selbst sein? Nichts an der Idee einer Menge scheint dies auszuschließen.
Wenn zum Beispiel \emph{alle} Mengen eine Sammlung von
Objekten bilden, könnte man meinen, dass sie zu einer einzigen Menge
zusammengefasst werden können---die Menge aller Mengen. 
Und da sie eine Menge ist, wäre sie Element der Menge aller Mengen. 

Das Russellsche Paradoxon entsteht, wenn wir die Eigenschaft betrachten,
sich selbst nicht als Element zu haben, \emph{nicht-selbst-enthaltend} zu sein.
Was wäre, wenn wir annehmen,
dass es eine Menge aller Mengen gibt, die sich selbst nicht als
Element haben? Existiert die folgende Menge?
\[
R = \Setabs{x}{x \notin x}
\]
 Wir können beweisen, dass dies nicht der Fall ist.

\begin{thm}[Russellsche Antinomie]\ollabel{thm:russells-paradox}
	Es gibt keine Menge $R = \Setabs{x}{x \notin x}$.
\end{thm}

\begin{proof}
Wenn $R = \Setabs{x}{x \notin x}$ existiert, dann
$R \in R$ genau dann, wenn $R \notin R$, was ein Widerspruch ist.
\end{proof}

\begin{tagblock}{novice}
\begin{explain}
Gehen wir diesen Beweis noch einmal langsam durch. Wenn $R$ existiert,
ergibt es Sinn zu fragen, ob $R \in
R$ ist oder nicht. Nehmen wir an, dass $R \in R$ tatsächlich gilt.
Nun, $R$~wurde definiert als die Menge aller
Mengen, die nicht Elemente ihrer selbst sind. Also, wenn $R \in R$,
dann hat $R$ selbst nicht die definierende Eigenschaft von $R$. Aber nur Mengen,
die diese Eigenschaft haben, sind in~$R$, also kann $R$ kein Element
von~$R$ sein, d.h. $R \notin R$. Aber $R$ kann nicht sowohl sein als auch nicht
Element aus~$R$ sein, also liegt ein Widerspruch vor.

Da die Annahme, dass $R \in R$ ist, zu einem Widerspruch führt, gilt
$R \notin R$. Aber auch das führt zu einem Widerspruch! Denn wenn $R
\notin R$, dann hat $R$ selbst die definierende Eigenschaft von $R$, und somit wäre $R$
Element von $R$, genau wie alle anderen nicht-selbst-enthaltenden Mengen.
Und noch einmal, es kann nicht sowohl Element als auch kein Element von~$R$ sein.
\end{explain}
\end{tagblock}

\begin{digress}
Wie kann man eine Mengenlehre aufstellen, die nicht in das
Russell-Paradoxon verfällt, d.h. die es vermeidet, die \emph{inkonsistente}
Behauptung aufzustellen, dass $R = \Setabs{x}{x \notin x}$ existiert?
Nun, wir müssten
Axiome aufstellen, die uns sehr präzise Bedingungen für die Aussage geben, wann
Mengen existieren (und wann sie nicht existieren). 
	
Die in diesem Kapitel skizzierte Mengenlehre tut das nicht. Sie ist
\emph{genuin na\"iv}. Sie sagt nur, dass Mengen der Extensionalität gehorchen
und dass man, wenn man einige Mengen hat, deren
Vereinigung, Schnittmenge usw. bilden kann. Es ist möglich, die Mengenlehre noch
strenger als hier zu entwickeln. \oliflabeldef{cumul:::part}{Diese Strenge wird
für den Teil \olref[cumul][][]{Teil} reserviert. Für den Moment werden wir
naiv vorgehen und vorsichtig versuchen, Widersprüche zu umgehen.}{}
\end{digress}


\end{document}
