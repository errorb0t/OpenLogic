% Teil: sets-functions-relations
% Kapitel: Mengen
% Abschnitt: pairs-and-products

\documentclass[../../../include/open-logic-section]{subfiles}

\begin{document}

\olfileid{sfr}{set}{pai}
\olsection{Paare, Tupel, kartesische Produkte}

\begin{explain}
Aus der Extensionalität folgt, dass Elemente von Mengen ungeordnet sind.
Wenn wir also eine Ordnung darstellen wollen, verwenden wir \emph{geordnete
Paare} $\tuple{x, y}$. Bei einem ungeordneten Paar $\{x, y\}$ spielt die Reihenfolge
keine Rolle: $\{x, y\} = \{y, x\}$. Bei einem geordneten Paar spielt sie eine Rolle: Wenn $x
\neq y$, dann ist $\tuple{x, y} \neq \tuple{y, x}$.

Wie sollten wir geordnete Paare mengentheoretisch betrachten?
Entscheidend ist, dass wir die Idee bewahren,
dass geordnete Paare identisch sind, wenn sie
das gleiche erste Element und das gleiche zweite Element haben, d.h.:
\[
  \tuple{a, b}= \tuple{c, d}\text{ gdw sowohl }a = c \text{ als auch }b=d.
\]
Wir können geordnete Paare mengentheoretisch mit Hilfe der Wiener-Kuratowski
Definition beschreiben.
\end{explain}

\begin{defn}[Geordnetes Paar]\ollabel{wienerkuratowski}
	$\tuple{a, b} = \{\{a\}, \{a, b\}\}$.
\end{defn}

\begin{prob}
	Beweisen Sie mit \olref[sfr][set][pai]{wienerkuratowski}, dass $\tuple{a,
	b}= \tuple{c, d}$ genau dann, wenn sowohl $a = c$ als auch $b=d$ gelten.
\end{prob}

\begin{explain}
Nachdem wir eine Definition eines geordneten Paares festgelegt haben, 
können wir sie zur Definition
weiterer Mengen verwenden. Zum Beispiel wollen wir manchmal auch geordnete Folgen von
mehr als zwei Objekten, z.B. \emph{Tripeln} $\tuple{x, y, z}$,
\emph{Quadrupeln} $\tuple{x, y, z, u}$, und so weiter gebrauchen.  Wir können die
Tripel als spezielle geordnete Paare betrachten, wobei das erste Element selbst ein
geordnetes Paar ist: $\tuple{x, y, z}$ ist $\tuple{\tuple{x, y},z}$. Dasselbe
gilt für Quadrupel: $\tuple{x,y,z,u}$ ist
$\tuple{\tuple{\tuple{x,y},z},u}$, und so weiter. Im Allgemeinen sprechen wir von
\emph{geordneten $n$-Tupeln} $\tuple{x_1, \dots, x_n}$.

Bestimmte Mengen von geordneten Paaren oder anderen geordneten $n$-Tupeln werden sich als nützlich erweisen.
\end{explain}

\begin{defn}[Kartesisches Produkt]
Bei gegebenen Mengen $A$ und $B$ ist ihr \emph{Kartesisches Produkt} $A \times B$
definiert durch
\[
  A \times B = \Setabs{\tuple{x, y}}{x \in A \text{ und } y \in B}.
\]
\end{defn}

\begin{ex}
Wenn $A = \{0, 1\}$, und $B = \{1, a, b\}$, dann ist ihr Produkt
\[
A \times B = \{ \tuple{0, 1}, \tuple{0, a}, \tuple{0, b},
    \tuple{1, 1}, \tuple{1, a}, \tuple{1, b} \}.
\]
\end{ex}

\begin{ex}
Ist $A$ eine Menge, so wird das Produkt von $A$ mit sich selbst, $A \times A$, auch
als~$A^2$ bezeichnet. Es ist die Menge \emph{aller} Paare $\tuple{x, y}$ mit
$x, y \in A$. Die Menge aller Tripel $\tuple{x, y, z}$ ist $A^3$,
und so weiter. Wir können eine rekursive Definition angeben:
\begin{align*}
  A^1 & = A\\
  A^{k+1} & = A^k \times A
\end{align*}
\end{ex}

\begin{prob}
Listen Sie alle Elemente von $\{1, 2, 3\}^3$ auf.
\end{prob}

\begin{prop}\ollabel{cardnmprod}
Wenn $A$ aus $n$ Elementen besteht und $B$ aus $m$ Elementen, dann hat $A
\times B$ genau $n\cdot m$ Elemente.
\end{prop}

\begin{proof}
Für jedes Element~$x$ in~$A$ gibt es $m$ Elemente der
Form $\tuple{x, y} \in A \times B$. Sei $B_x = \Setabs{\tuple{x, y}}{y
  \in B}$. Da immer, wenn $x_1 \neq x_2$, auch $\tuple{x_1, y} \neq
\tuple{x_2, y}$ gilt, folgt $B_{x_1} \cap B_{x_2} = \emptyset$. Wenn aber $A = \{x_1,
\dots, x_n\}$, dann ist $A \times B = B_{x_1} \cdot \dots \cdot B_{x_n}$, und hat somit
$n\cdot m$ Elemente.

Um das zu veranschaulichen, ordne die Elemente von~$A \times B$ in einer Tabelle an:
\[
\begin{array}{rcccc}
  B_{x_1} = & \{\tuple{x_1, y_1} & \tuple{x_1, y_2} & \dots & \tuple{x_1, y_m}\} \\
  B_{x_2} = & \{\tuple{x_2, y_1} & \tuple{x_2, y_2} & \dots & \tuple{x_2, y_m}\} \\
  \vdots & & \vdots\\
  B_{x_n} = & \{\tuple{x_n, y_1} & \tuple{x_n, y_2} & \dots & \tuple{x_n, y_m}\} \\
\end{array}
\]
Da die $x_i$ alle verschieden sind, und die $y_j$ alle verschieden sind, sind keine
zwei der Paare in dieser Tabelle gleich, und es gibt $n\cdot m$
von ihnen.
\end{proof}

\begin{prob}
Zeigen Sie mittels Induktion auf~$k$, dass für alle $k \ge 1$gilt: Wenn $A$ aus $n$ Elementen besteht,
dann hat $A^k$ genau $n^k$ Elemente.
\end{prob}

\begin{ex}
Wenn $A$ eine Menge ist, dann ist ein \emph{Wort} über~$A$ eine beliebige Folge von Elementen
aus~$A$. Eine Folge kann man sich als ein $n$-Tupel von
Elementen aus~$A$ vorstellen. Wenn zum Beispiel $A = \{a, b, c\}$, dann kann die
Folge ``$bac$'' als das Tripel~$\tuple{b, a, c}$ aufgefasst werden.
Wörter, d.h. Folgen von Symbolen, sind von entscheidender Bedeutung in der
Informatik. Als Konvention zählen wir Elemente von~$A$ als
Folgen der Länge~$1$, und $\emptyset$ als Folge der Länge~$0$.
Die Menge \emph{aller} Wörter über~$A$ ist dann
\[
A^* = \{\emptyset\} \cup A \cup A^2 \cup A^3 \cup \dots
\]
\end{ex}

\end{document}
