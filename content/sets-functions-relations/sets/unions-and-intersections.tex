% Teil: sets-functions-relations
% Kapitel: sets
% Abschnitt: unions-and-intersections

\documentclass[../../../include/open-logic-section]{subfiles}

\begin{document}

\olfileid{sfr}{set}{uni}
\olsection{Vereinigung und Durchschnitt}

\begin{explain}
In \olref[sfr][set][bas]{sec} haben wir abstrakte Mengendefinitionen eingeführt,
 d.h. Definitionen der Form $\Setabs{x}{\phi(x)}$.
Hier berufen wir uns auf eine Eigenschaft~$\phi$. Diese Eigenschaft kann
Mengen, die wir bereits definiert haben, betreffen. Wenn also zum Beispiel $A$ und~$B$ Mengen sind,
besteht die Menge $\Setabs{x}{x \in A \lor x \in B}$ aus denjenigen
Objekten, die Elemente von entweder $A$ oder~$B$ sind, d.h. es ist die
Menge, die die Elemente von $A$ und~$B$ zusammenfasst. Wir können uns das vorstellen
wie in \olref{fig:union}, wobei der schraffierte Bereich die
Elemente anzeigt, die in beiden Mengen $A$ und~$B$ liegen.

\begin{figure}
  \olasset{assets/diagrams/union.tikz}
  \caption{Die Vereinigung $A \cup B$ zweier Mengen ist die Menge der Elemente von
   $A$ zusammen mit denen von~$B$.}
  \ollabel{fig:union} 
\end{figure}

Diese Operation auf Mengen - das Kombinieren von Mengen - ist sehr nützlich und üblich,
und deshalb geben wir ihr einen formalen Namen und ein Symbol. 
\end{explain}

\begin{defn}[Vereinigung]
Die \emph{Vereinigung} von zwei Mengen $A$ und $B$, geschrieben $A \cup B$, ist die
Menge aller Dinge, die Elemente aus~$A$ oder aus~$B$ oder aus beiden Mengen sind.
\[
A \cup B = \Setabs{x}{x \in A \lor x \in B}
\]
\end{defn}

\begin{ex}
Da das mehrmalige Auftreten von Elementen keine Rolle spielt, enthält die Vereinigung von zwei
Mengen, die ein Element gemeinsam haben, dieses Element nur einmal,
z.B., $\{ a, b, c\} \cup \{ a, 0, 1\} = \{a, b, c, 0, 1\}$.

Die Vereinigung einer Menge und einer ihrer Teilmengen ist einfach die größere Menge: $\{a,
b, c \} \cup \{a \} = \{a, b, c\}$.

Die Vereinigung einer Menge mit der leeren Menge ist gleich der Menge selbst: $\{a,
b, c \} \cup \emptyset = \{a, b, c \}$.
\end{ex}

\begin{prob}
Beweise, dass wenn $A \subseteq B$, dann $A \cup B = B$.
\end{prob}

\begin{explain}
Wir können auch eine \glqq duale\grqq{} Operation zur Vereinigung betrachten. Dies ist die
Operation, die die Menge aller Elemente bildet, die sowohl Elemente
von~$A$ als auch von~$B$ sind. Diese Operation wird als 
\emph{Durchschnitt} oder schlicht \emph{Schnitt} bezeichnet, und kann wie in \olref{fig:intersection} dargestellt werden.
\begin{figure}
  \olasset{assets/diagrams/intersection.tikz}
  \caption{Die Schnittmenge $A \cap B$ von zwei Mengen ist die Menge der
    Elemente, die sie sich teilen.}
  \ollabel{fig:intersection}
\end{figure}
\end{explain}

\begin{defn}[Schnittmenge]
Die \emph{Schnittmenge} zweier Mengen $A$ und $B$, geschrieben $A \cap B$, ist
die Menge aller Dinge, die Elemente sowohl von $A$ als auch von~$B$ sind.
\[
A \cap B = \Setabs{x}{x \in A \land x \in B}
\]
Zwei Mengen werden \emph{disjunkt} genannt, wenn ihre Schnittmenge
leer ist. Das bedeutet, dass sie keine Elemente gemeinsam haben.
\end{defn}

\begin{ex}
Wenn zwei Mengen keine Elemente gemeinsam haben, ist ihre Schnittmenge leer:
$\{ a, b, c\} \cap \{ 0, 1\} = \emptyset$.

Wenn zwei Mengen aber Elemente gemeinsam haben, ist ihre Schnittmenge die Menge
aller dieser Elemente:\\ $\{a, b, c \} \cap \{a, b, d \} = \{a, b\}$.

Die Schnittmenge einer Menge mit einer ihrer Teilmengen ist einfach die kleinere
Menge: $\{a, b, c\} \cap \{a, b\} = \{a, b\}$.

Die Schnittmenge jeder Menge mit der leeren Menge ist leer: $\{a, b, c \}
\cap \emptyset = \emptyset$.
\end{ex}

\begin{prob}
Beweisen Sie lückenlos, dass wenn $A \subseteq B$, dann $A \cap B = A$.
\end{prob}

\begin{explain}
Wir können auch die Vereinigung oder die Schnittmenge von mehr als zwei
Mengen bilden. Eine elegante Art, damit im Allgemeinen umzugehen, ist die Folgende:
Nehmen wir an, wir fassen alle Mengen, über die wir die Vereinigung (oder den Durchschnitt) bilden wollen,
in einer einzigen Menge zusammen. Dann können wir die Vereinigung
aller unserer ursprünglichen Mengen als die Menge aller Objekte definieren, die zu mindestens einem
mindestens einem Element der Menge gehören, und die Schnittmenge als die Menge der
allen Objekten, die zu jedem Element der Menge gehören.
\end{explain}

\begin{defn}
Wenn $A$ eine Menge von Mengen ist, dann ist $\bigcup A$ die Menge der Elemente von
Elementen aus~$A$:
\begin{align*}
\bigcup A & = \Setabs{x}{x \text{ gehört zu Element aus } A},
\text{ d.h., }\\
& = \Setabs{x}{\text{es gibt ein } B \in A
  \text{ sodass } x \in B}.
\end{align*}
\end{defn}

\begin{defn}
Wenn $A$ eine Menge von Mengen ist, dann ist $\bigcap A$ die Menge der Objekte, die
alle Elemente aus~$A$ gemeinsam haben:
\begin{align*}
\bigcap A & = \Setabs{x}{x \text{ gehört zu jedem Element aus } A},
\text{ d.h., }\\
 & = \Setabs{x}{\text{für alle } B \in A \text{ gilt } x \in B}.
\end{align*}
\end{defn}

\begin{ex}
Angenommen $A = \{ \{ a, b \}, \{ a, d, e \}, \{ a, d \} \}$.
Dann sind $\bigcup A = \{ a, b, d, e \}$ und $\bigcap A = \{ a \}$.
\end{ex}
\begin{prob}
	Zeigen Sie, dass wenn $A$ eine Menge ist und $A \in B$, dann gilt $A \subseteq \bigcup B$.
\end{prob}

Das Gleiche könnte man auch für eine Folge von Mengen $A_1$, $A_2$, \dots tun.
\begin{align*}
\bigcup_i A_i & = \Setabs{x}{x \text{ gehört zu einer der } A_i}\\
\bigcap_i A_i & = \Setabs{x}{x \text{ gehört zu jedem } A_i}.
\end{align*}

Wenn wir einen \emph{Index} von Mengen haben, d.h. eine Menge $I$, für die wir
$A_i$ für jedes $i \in I$ betrachten, können wir auch diese
Abkürzungen verwenden:
\begin{align*}
	\bigcup_{i \in I} A_i & = \bigcup \Setabs{A_i }{i \in I}\\
	\bigcap_{i \in I} A_i & = \bigcap\Setabs{A_i}{i \in I}
\end{align*}

Zuletzt können wir noch über die Menge aller Elemente aus~$A$ nachdenken,
die nicht aus~$B$ sind. Wir können das wie in \olref{difference} darstellen.

\begin{figure}
  \olasset{assets/diagrams/difference.tikz}
  \caption{Die Differenz $A \setminus B$ zweier Mengen ist die Menge der
    derjenigen Elemente aus~$A$, die nicht auch Elemente aus~$B$ sind.}
  \ollabel{difference}
\end{figure}

\begin{defn}[Differenz]
Die \emph{Differenzmenge}~$A \setminus B$ ist die Menge aller Elemente von
$A$, die nicht auch Elemente von~$B$ sind, d.h.,
\[
A\setminus B = \Setabs{x}{x\in A \text{ und } x \notin B}.
\]
\end{defn}

\begin{prob}
	Beweisen Sie, dass, wenn $A \subsetneq B$, dann $B \setminus A \neq \emptyset$ gilt.
\end{prob}

\end{document}