% Teil: Mengen-Funktionen-Relationen
% Kapitel: Relationen
% Abschnitt: relations-as-sets

\documentclass[../../../include/open-logic-section]{subfiles}

\begin{document}

\olfileid{sfr}{rel}{set}
\olsection{Relationen als Mengen}

\begin{explain}
In \olref[sfr][set][imp]{sec} haben wir einige wichtige Mengen erwähnt:
$\Nat$, $\Int$, $\Rat$, $\Real$. Wir erinnern uns an manche bestimmten Relationen
zwischen den Elementen einiger dieser Mengen.
Zum Beispiel gibt es eine ganz normale \emph{Ordnungsrelation} auf jeder dieser Mengen. 
Es gibt auch die Relation \emph{ist identisch mit},
in der jedes Objekt zu sich selbst und zu keinem anderen Ding steht. Es gibt
viele weitere interessante Relationen, die wir kennenlernen werden, und noch mehr
mögliche Relationen. Bevor wir sie jedoch betrachten, wollen wir zunächst
darauf hinweisen, dass wir Relationen als eine besondere Art von Mengen betrachten können. 
  
Dazu erinnern wir uns an zwei Dinge aus \olref[sfr][set][pai]{sec}. Erstens
erinnern wir uns an den Begriff \emph{geordnetes Paar}: Für $a$ und $b$ können wir
Tupel ${a, b}$ bilden. Wichtig ist, dass die Reihenfolge der Elemente hier \emph{doch}
eine Rolle spielt. Wenn also $a \neq b$, dann $\tuple{a, b} \neq \tuple{b, a}$.
(Im Gegensatz zu ungeordneten Paaren, d.h. $2$-Elemente-Mengen, bei denen
$\{a, b\}=\{b, a\}$.) Zweitens: Erinnern wir uns an den Begriff \emph{kartesisches
Produkt}: Wenn $A$ und $B$ Mengen sind, dann können wir~$A \times B$ bilden, die
Menge aller Paare $\tuple{x, y}$ mit $x \in A$ und $y \in B$.
Insbesondere ist $A^{2}= A \times A$ die Menge aller geordneten Paare
aus~$A$.

Wir nehmen nun eine besondere Relation auf einer Menge in den Blick: die $<$-Relation
auf der Menge~$\Nat$ der natürlichen Zahlen. Betrachten wir die Menge aller Paare von
Zahlen $\tuple{n, m}$ mit $n<m$, d.h.,
\[
  R=\Setabs{\tuple{n, m}}{n, m \in \Nat \text{ und } n<m}.
\]
Es besteht ein enger Zusammenhang zwischen $n$, das kleiner ist als $m$, und dem
Paar $\tuple{n, m}$, das ein Element aus $R$ ist, nämlich:
\[
      n<m\text{ gdw }\tuple{n, m} \in R.
\]
In der Tat können wir ohne Informationsverlust die Menge $R$ als
\emph{die $<$-Relation auf $\Nat$ selbst} betrachten. 

Auf dieselbe Weise können wir eine Teilmenge von $\Nat^{2}$ für jede
Beziehung zwischen Zahlen konstruieren. Umgekehrt kann man für eine beliebige Menge von Paaren von
Zahlen $S \subseteq \Nat^{2}$, gibt es eine entsprechende Beziehung
zwischen Zahlen, nämlich die Beziehung $n$ zu $m$ genau dann, wenn $\tuple{n, m} \in S$.
Dies rechtfertigt die folgende Definition:
\end{explain}
  
\begin{defn}[Binäre Relation] 
Eine \emph{binäre Relation} auf einer Menge $A$ ist eine Teilmenge von~$A^{2}$. Wenn $R
\subseteq A^{2}$ eine binäre Relation auf~$A$ und $x, y \in A$ ist, schreibt man kurz
$Rxy$ (oder $xRy$) für $\tuple{x, y} \in R$.
\end{defn}
  
\begin{ex}
  \ollabel{relations}
Die Menge $\Nat^{2}$ der Paare natürlicher Zahlen lässt sich in einer
zweidimensionalen Matrix wie folgt auflisten:
\[
  \begin{array}{ccccc}
  \mathbf{\tuple{ 0,0 }} & \tuple{ 0,1 } &
    \tuple{ 0,2 } & \tuple{ 0,3 } & \ldots\\
  \tuple{ 1,0 } & \mathbf{\tuple{ 1,1 }} &
    \tuple{ 1,2 } & \tuple{ 1,3 } & \ldots\\
  \tuple{ 2,0 } & \tuple{ 2,1 } &
    \mathbf{\tuple{ 2,2 }} & \tuple{ 2,3 } & \ldots\\
  \tuple{ 3,0 } & \tuple{ 3,1 } & \tuple{ 3,2 } &
    \mathbf{\tuple{ 3,3 }} & \ldots\\
  \vdots & \vdots & \vdots & \vdots & \mathbf{\ddots}
  \end{array}
\]
Wir haben die Diagonale hier fett gesetzt, da die Teilmenge von $\Nat^2$
bestehend aus den Paaren, die auf der Diagonale liegen, d.h.,
\[
  \{\tuple{0,0 }, \tuple{ 1,1 }, \tuple{ 2,2 }, \dots\},
  \]
die \emph{Identitätsrelation auf}~$\Nat$ ist. (Da die Identitätsrelation sehr
beliebt ist, definieren wir $\Id{A}=\Setabs{\tuple{ x,x }}{x \in
A}$ für jede Menge $A$.) Die Teilmenge aller Paare, die oberhalb der
Diagonale liegen, d.h.,
\[
  L = \{\tuple{ 0,1 },\tuple{ 0,2 },\ldots,\tuple{ 1,2 },
  \tuple{ 1,3 }, \dots, \tuple{ 2,3 }, \tuple{ 2,4 },\ldots\},
\]
ist die \emph{kleiner-als}-Relation, d.h. $Lnm$ gdw $n<m$. Die Teilmenge der
Paare unterhalb der Diagonale, d.h.,
\[
  G=\{ \tuple{ 1,0 },\tuple{ 2,0 },\tuple{
    2,1 }, \tuple{ 3,0 },\tuple{ 3,1 },\tuple{ 3,2 }, \dots\},
\]
ist die \emph{größer-als}-Relation, d.h. $Gnm$ gdw $n>m$. Die Vereinigung
von $L$ mit $I$, die wir $K=L\cup I$ nennen können, ist die \emph{kleiner-gleich}-Relation:
$Knm$ gdw $n \le m$. In ähnlicher Weise ist $H=G \cup
I$ die \emph{größer-gleich}-Relation. Diese Relationen
$L$, $G$, $K$ und $H$ sind besondere Arten von Relationen, genannt
\emph{Ordnungen}. $L$ und $G$ haben die Eigenschaft, dass keine Zahl zu sich selbst in der Relation
$L$ oder $G$ steht (d.h. für alle $n$ weder $Lnn$ noch $Gnn$).
Relationen mit dieser Eigenschaft nennt man \emph{irreflexiv}, und wenn
sie zufällig auch Ordnungen sind, nennt man sie \emph{Striktordnungen}.
\end{ex}

\begin{explain}
Obwohl Ordnungen und Identität wichtige und natürliche Beziehungen sind,
sollte hervorgehoben werden, dass nach unserer Definition \emph{jede}
Teilmenge von $A^{2}$ eine Relation auf~$A$ ist, unabhängig davon, wie unnatürlich oder
konstruiert sie erscheint. Insbesondere ist $\emptyset$ eine Relation auf jeder
Menge (die \emph{leere Relation}, die kein Paar von Elementen trägt), und
$A^{2}$~selbst ist ebenfalls eine Relation auf~$A$ (eine, die jedes Paar
trägt), genannt die \emph{universelle Relation}. Aber auch etwas wie
$E=\Setabs{\tuple{n, m}}{n>5 \text{ oder } m \times n \ge 34}$ zählt als
eine Relation.
\end{explain}
  
\begin{prob}
  Listen Sie die Elemente der Relation $\subseteq$ auf der Menge
  $\Pow{\{a, b, c\}}$ auf.
\end{prob}

\end{document}
