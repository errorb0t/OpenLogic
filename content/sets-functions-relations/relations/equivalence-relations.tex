% Teil: Mengen-Funktionen-Relationen
% Kapitel: Beziehungen
% Abschnitt: Äquivalenz-Relationen

\documentclass[../../../include/open-logic-section]{subfiles}

\begin{document}

\olfileid{sfr}{rel}{eqv}

\olsection{Äquivalenzrelationen}

Die Identitätsrelation auf einer Menge ist reflexiv, symmetrisch und
transitiv. Relationen~$R$, die alle drei dieser Eigenschaften besitzen, 
kommen oft vor.

\begin{defn}[Äquivalenzrelation] 
Eine Relation ${R \subseteq A^2}$, die reflexiv, symmetrisch und
transitiv ist, nennt man eine \emph{Äquivalenzrelation}. Elemente $x$
und $y$ von~$A$ heißen \emph{äquivalent unter $R$}, falls~$Rxy$.
\end{defn}

Aus Äquivalenzbeziehungen ergibt sich der Begriff der \emph{Äquivalenzklasse}.
Eine Äquivalenzrelation \glqq zerlegt\grqq{} den Definitionsbereich in
verschiedene Partitionen. Innerhalb jeder Partition stehen alle Objekte zueinander 
in Beziehung; und keine Objekte aus verschiedenen Partitionen
stehen in Beziehung zueinander. Manchmal ist es hilfreich, einfach über
diese Partitionen direkt zu sprechen. Zu diesem Zweck führen wir eine
Definition ein:

\begin{defn}\ollabel{def:Äquivalenzklasse}
Sei $R \subseteq A^2$ eine Äquivalenzrelation. Für jedes $x \in A$
ist die \emph{Äquivalenzklasse} von $x$ in~$A$ die Menge ${\equivrep{x}{R}
= \Setabs{y \in A}{Rxy}}$. Die \emph{Quotientenmenge} von $R$ auf~$A$ ist
$\equivclass{A}{R} = \Setabs{\equivrep{x}{R}}{x \in A}$, d.h. die Menge
dieser Äquivalenzklassen. 
\end{defn}

Die nächste Aussage bestätigt unsere Definition einer Äquivalenzklasse,
indem sie beweist, dass die Äquivalenzklassen tatsächlich die Partitionen von~$A$ sind:

\begin{prop}
Falls $R \subseteq A^2$ eine Äquivalenzrelation ist, dann gilt $Rxy$ gdw
$\equivrep{x}{R} = \equivrep{y}{R}$.
\end{prop}

\begin{proof}
Für die hin-Richtung sei $Rxy$, und $z \in
\equivrep{x}{R}$. Nach Definition ist dann $Rxz$. Da $R$ eine
Äquivalenzrelation ist, gilt $Ryz$. (Um genau zu sein: Da $Rxy$ gilt und~$R$
symmetrisch ist, wissen wir $Ryx$, und da $Rxz$ gilt und~$R$ transitiv ist,
haben wir~$Ryz$). Also $z \in \equivrep{y}{R}$. Allgemeiner:
$\equivrep{x}{R} \subseteq \equivrep{y}{R}$. Aber genauso gilt:
$\equivrep{y}{R} \subseteq \equivrep{x}{R}$. Also $\equivrep{x}{R} =
\equivrep{y}{R}$, wegen der Extensionalität.

Für die rück-Richtung nehmen wir an, dass $\equivrep{x}{R} =
\equivrep{y}{R}$. Da $R$ reflexiv ist, ist $Ryy$, also $y \in
\equivrep{y}{R}$. Also ist auch $y \in \equivrep{x}{R}$ aufgrund der Annahme,
dass $\equivrep{x}{R} = \equivrep{y}{R}$. Also $Rxy$.
\end{proof}

\begin{ex}
Ein schönes Beispiel für Äquivalenzbeziehungen stammt aus der Zahlentheorie.
Definiere für beliebige $a$, $b$ und $n \in \Nat$, dass $a \equiv_n b$ gdw
die Division von $a$ durch~$n$ denselben Rest ergibt wie die Division von $b$ durch~$n$.
(Etwas symbolischer: $a \equiv_n b$ gdw für ein $k \in
\Int$ gilt: $a - b = kn$.) Nun ist $\equiv_n$ für beliebige~$n$ eine Äquivalenzrelation.
Und es gibt genau $n$ verschiedene Äquivalenzklassen,
die durch~$\equiv_n$ erzeugt werden; das heißt, $\equivclass{\Nat}{\equiv_n}$ hat
$n$ Elemente. Diese sind: die Menge der durch $n$ teilbaren Zahlen
ohne Rest, d.h. $\equivrep{0}{\equiv_n}$; die Menge der Zahlen
teilbar durch $n$ mit Rest~$1$, d.h. $\equivrep{1}{\equiv_n}$;
\ldots; und die Menge der Zahlen, die durch~$n$ mit Rest~$n-1$ teilbar sind,
d.h.,~$\equivrep{n-1}{\equiv_n}$.
\end{ex}

\begin{prob}
Zeigen Sie, dass $\equiv_n$ für jedes $n \in
\Nat$ eine Äquivalenzrelation ist und, dass $\equivclass{\Nat}{\equiv_n}$
genau $n$ Elemente hat.
\end{prob}

\end{document}