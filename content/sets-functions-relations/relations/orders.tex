% Teil: Mengen-Funktionen-Relationen
% Kapitel: Relationen
% Abschnitt: orders

\documentclass[../../../include/open-logic-section]{subfiles}

\begin{document}

\olfileid{sfr}{rel}{ord}
\olsection{Ordnungen}

\begin{explain}
Bei vielen unserer Vergleiche geht es darum, einige Objekte als
\glqq weniger als\grqq{}, \glqq gleich\grqq{} oder \glqq größer als\grqq{}
andere Objekte in einer bestimmten Hinsicht zu beschreiben.
Dabei geht es um \emph{Ordnungsrelationen}. Aber es gibt
verschiedene Arten von Ordnungsrelationen. Einige erfordern zum Beispiel, dass
dass zwei Objekte vergleichbar sind, bei anderen ist das nicht der Fall. Einige schließen die Identität
ein (wie~$\le$) und manche schließen sie aus (wie~$<$). Es wird sich als nützlich erweisen,
sie zu klassifizieren.
\end{explain}

\begin{defn}[Quasiordnung]
Eine Relation, die sowohl reflexiv als auch transitiv ist, nennt man eine
\emph{Quasiordnung}. 
\end{defn}

\begin{defn}[Halbordnung]
Eine Quasiordnung, die auch antisymmetrisch ist, nennt man eine
\emph{Halbordnung}.
\end{defn}

\begin{defn}[Totalordnung]
Eine Halbordnung, die auch verbunden ist, nennt man eine
\emph{Totalordnung} oder \emph{lineare Ordnung}.
\end{defn}

\begin{ex}
Jede lineare Ordnung ist auch eine partielle Ordnung, und jede partielle Ordnung ist
auch eine Quasiordnung, aber die Umkehrungen sind im Allgemeinen nicht gültig.
Die universelle Relation
auf~$A$ ist eine Quasiordnung, da sie reflexiv und transitiv ist. Aber, wenn
$A$ mehr als ein Element hat, ist die universelle Relation nicht
antisymmetrisch und somit keine Halbordnung.
\end{ex}

\begin{ex}
Betrachten Sie die \emph{nicht-länger-als}-Relation $\preccurlyeq$
auf~$\Bin^*$: $x \preccurlyeq y$ gdw $\len{x} \le \len{y}$. Dies ist eine
Quasiordnung (reflexiv und transitiv), und sogar verbunden, aber keine
Halbordnung, da sie nicht antisymmetrisch ist. Zum Beispiel ist $01
\preccurlyeq 10$ und $10 \preccurlyeq 01$, aber $01 \neq 10$.
\end{ex}

\begin{ex}
Eine wichtige Halbordnung ist die Relation $\subseteq$ auf einer Menge von
Mengen. Diese ist im Allgemeinen keine lineare Ordnung, denn betrachten wir für $a \neq b$ die Menge
$\Pow{\{a, b\}} = \{\emptyset, \{a\}, \{b\}, \{a,b\}\}$,
sehen wir, dass $\{a\} \nsubseteq \{b\}$ und $\{a\} \neq \{b\}$ und $\{b\}
\nsubseteq \{a\}$.
\end{ex}

\begin{ex}
Die Beziehung von \emph{Teilbarkeit ohne Rest} gibt uns eine
Halbordnung, die keine Totalordnung ist. Für ganze Zahlen $n$ und $m$ schreiben wir
$n\mid m$, um zu sagen, dass $m$ von $n$ (gleichmäßig) geteilt wird, d.h., wenn es
irgendeine ganze Zahl~$k$ gibt, sodass $m=kn$. Für $\Nat$ ist dies eine partielle Ordnung, aber keine
lineare Ordnung: zum Beispiel $2\nmid3$ und auch $3\nmid2$. Betrachtet man
als Relation auf $\Int$ betrachtet, ist die Teilbarkeit nur eine Vorordnung, da
sie nicht antisymmetrisch ist: $1\mid-1$ und $-1\mid1$ aber $1\neq-1$.
\end{ex}

\begin{defn}[Striktordnung]
Ein \emph{Striktordnung}, auch \emph{strenge Ordnung}, ist eine Beziehung, die irreflexiv, asymmetrisch,
und transitiv ist.
\end{defn}

\begin{defn}[Strenge Totalordnung]{def:strictlinearorder}
Eine strenge Ordnung, die auch verbunden ist, nennt man eine
\emph{strenge Totalordnung} (oder \emph{strenge lineare Ordnung}).
\end{defn}

\begin{ex}
$\le$ ist die lineare Ordnung, die der strengen linearen
Ordnung~$<$ entspricht. $\subseteq$ ist die partielle Ordnung, die der
strengen Ordnung~$\subsetneq$ entspricht.
\end{ex}

Jede Striktordnung $R$ auf~$A$ kann folgendermaßen in eine Halbordnung umgewandelt werden:
Der Relation wird die Diagonale $\Id{A}$, d.h. alle Paare~$\tuple{x,
x}$ hinzugefügt.  (Dies nennt man die \emph{reflexive Hülle} von~$R$.)
Umgekehrt kann man aus einer Halbordnung eine Striktordnung machen,
indem man~$\Id{A}$ entfernt. Die nächsten beiden Aussagen präzisieren das.
\begin{prop}\ollabel{prop:stricttopartial}
Falls $R$ eine Striktordnung auf~$A$ ist, dann ist $R^+ = R \cup \Id{A}$ eine
Halbordnung. Außerdem gilt: Falls $R$ streng linear ist, dann ist $R^+$ eine lineare
Ordnung.
\end{prop}

\begin{proof}
Angenommen, $R$ ist eine Striktordnung, d.h. $R \subseteq A^2$ und $R$ ist
irreflexiv, asymmetrisch und transitiv. Sei $R^+ = R \cup \Id{A}$. Wir
müssen zeigen, dass $R^+$ reflexiv, antisymmetrisch und transitiv ist.

$R^+$ ist offenbar reflexiv, da $\tuple{x, x} \in \Id{A} \subseteq
R^+$ für alle $x \in A$. 

Um zu zeigen, dass $R^+$ antisymmetrisch ist, nehmen wir für einen Widerspruchsbeweis an,
dass $R^+xy$ und $R^+yx$, aber $x \neq y$.
Da $\tuple{x,y} \in R \cup \Id{X}$, aber
$\tuple{x, y} \notin \Id{X}$ ist, gilt $\tuple{x, y} \in R$, d.h.,
$Rxy$. Analog folgt $Ryx$. Das widerspricht aber der Annahme, dass
dass $R$ asymmetrisch ist.

Um die Transitivität zu beweisen, nehmen wir an, dass $R^+xy$ und $R^+yz$. Wenn sowohl
$\tuple{x, y} \in R$ als auch $\tuple{y,z} \in R$, dann ist $\tuple{x, z} \in
R$, da $R$~ transitiv ist. Andernfalls ist entweder $\tuple{x, y} \in
\Id{X}$, d.h. $x = y$, oder $\tuple{y, z} \in \Id{X}$, d.h., $y = z$.
Im ersten Fall gilt $R^+yz$ nach der Voraussetzung $x = y$, also
$R^+xz$. Analog folgt der zweite Fall. In beiden Fällen ist also $R^+xz$,
also ist $R^+$ transitiv.

In Bezug auf den \glqq Außerdem\grqq{}-Nachsatz: Nehmen wir an, dass $R$ eine strenge lineare Ordnung ist,
d.h., dass $R$ verbunden ist. Für alle $x \neq y$ ist also entweder $Rxy$
oder~$Ryx$, d.h. entweder $\tuple{x, y} \in R$ oder $\tuple{y, x} \in R$.
Da $R \subseteq R^+$ ist, gilt dies auch für $R^+$, also ist $R^+$
ebenfalls verbunden.
\end{proof}

\begin{prop}\ollabel{prop:partialtostrict}
Wenn $R$ eine Halbordnung auf~$X$ ist, dann ist $R^- = R \setminus \Id{X}$ eine
Striktordnung. Und wenn $R$ linear ist, dann ist $R^-$ streng linear.
\end{prop}

\begin{proof}
Dieser Beweis wird als Übung belassen.
\end{proof}

\begin{prob}
Führen Sie einen Beweis für \olref[sfr][rel][ord]{prop:partialtostrict}. 
\end{prob}

Das folgende einfache Ergebnis beweist, dass totale Ordnungen
eine extensionalitätsähnliche Eigenschaft erfüllen:

\begin{prop}\ollabel{prop:extensionality-totalorders}
Wenn $<$ total $A$ ordnet, dann gilt: 
\[
  (\forall a, b \in A)((\forall x \in A)(x < a \liff x < b) \lif a = b)
\]
\end{prop}

\begin{proof}
Angenommen $(\forall x \in A)(x < a \liff x < b)$. Wenn $a < b$, dann ist $a <
a$, was ein Widerspruch dazu ist, dass $<$ irreflexiv ist; also $a \nless b$.
Genauso gilt: $b \nless a$. Also $a = b$, da $<$ verbunden ist.
\end{proof}

\end{document}
