% Teil: Mengen-Funktionen-Relationen
% Kapitel: Relationen
% Abschnitt: Operationen

\documentclass[../../../include/open-logic-section]{subfiles}

\begin{document}

\olfileid{sfr}{rel}{ops}
\olsection{Operationen auf Relationen}

Es ist oft nützlich, Relationen zu ändern oder zu kombinieren. In
\olref[sfr][rel][ord]{prop:stricttopartial} haben wir die \emph{Vereinigung}
von Relationen betrachtet, die lediglich die Vereinigung zweier Relationen als
Mengen von Paaren ist. In ähnlicher Weise wurde in \olref[sfr][rel][ord]{prop:partialtostrict},
haben wir die relative Differenz von Relationen betrachtet. Hier sind einige
andere Operationen, die wir mit Relationen durchführen können.

\begin{defn}\ollabel{relationoperations} 
Seien $R$, $S$ Relationen und $A$ eine beliebige Menge. 

Das \emph{Umkehrrelation} von $R$ ist $R^{-1} = \Setabs{\tuple{y, x}}{\tuple{x,
    y} \in R}$.

Das \emph{relative Produkt}, auch \emph{Vorwärtsverkettung}, von $R$ und $S$ ist $(R \mid S) =
\{\tuple{x, z} : \exists y(Rxy \land Syz)\}$.

Die \emph{Einschränkung} von $R$ auf $A$ ist $\funrestrictionto{R}{A}= R
\cap A^2$.

Das \emph{Bild} von $A$ unter $R$ ist $\funimage{R}{A} = \{y :
(\exists x \in A)Rxy\}$.
\end{defn}

\begin{ex}
Sei $S \subseteq \Int^2$ die Nachfolgerelation auf~$\Int$, d.h.,
$S = \Setabs{\tuple{x, y} \in \Int^2}{x + 1 = y}$, so dass $Sxy$ gdw $x + 1 = y$.

$S^{-1}$ ist die Vorgängerrelation auf $\Int$, d.h.,
$\Setabs{\tuple{x,y}\in\Int^2}{x -1 =y}$.

$S\mid S$ ist 
$ \Setabs{\tuple{x,y}\in\Int^2}{x + 2 =y}$

$\funrestrictionto{S}{\Nat}$ ist die Nachfolgerelation auf~$\Nat$.

$\funimage{S}{\{1,2,3\}}$ ist $\{2, 3, 4\}$.
\end{ex}

\begin{defn}[Transitive Hülle]Sei $R \subseteq A^2$ eine binäre Relation. 
	
Die \emph{transitive Hülle} von~$R$ ist $R^+ = \bigcup_{0 < n \in
\Nat} R^n$, wobei wir rekursiv $R^1 = R$ und $R^{n+1} = R^n
\mid R$ definieren.

Die \emph{reflexive transitive Hülle} von $R$ ist $R^* = R^+ \cup
\Id{A}$.
\end{defn}

\begin{ex}
Betrachten wir die Nachfolgerelation $S \subseteq \Int^2$. $S^2xy$ gdw $x + 2 =
y$, $S^3xy$ gdw $x + 3 = y$, usw. Also $S^+xy$ gdw $x + n = y$ für beliebige
$n \geq 1$. Mit anderen Worten: $S^+xy$ gdw $x < y$, und $S^*xy$ gdw $x \le
y$.
\end{ex}

\begin{prob}
Zeigen Sie, dass die transitive Hülle von $R$ tatsächlich transitiv ist.
\end{prob}

\end{document}