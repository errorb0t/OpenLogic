% Teil: sets-functions-relations
% Kapitel: relations
% Abschnitt: special-properties

\documentclass[../../../include/open-logic-section]{subfiles}

\begin{document}

\olfileid{sfr}{rel}{prp}
\olsection{Besondere Eigenschaften von Relationen}

\begin{intro}
Einige Arten von Relationen sind so häufig, dass sie besondere Namen haben.
Zum Beispiel beziehen sich $\le$ und~$\subseteq$ beide auf
ihre jeweiligen Definitionsbereiche (z. B. $\Nat$ im Fall von~$\le$ und
$\Pow{A}$ im Fall von~$\subseteq$) in ähnlicher Weise.  Um herauszufinden,
wie ähnlich sich diese Beziehungen sind, und wie sie sich unterscheiden,
kategorisieren wir sie nach einigen speziellen Eigenschaften, die Relationen
haben können.  Es stellt sich heraus, dass (Kombinationen von) einiger dieser speziellen
Eigenschaften besonders wichtig sind: Ordnungen und Äquivalenzbeziehungen.
\end{intro}

\begin{defn}[Reflexivität]
Eine Relation $R \subseteq A^2$ heißt \emph{reflexiv} gdw für jedes $x \in
A$ gilt: $Rxx$.
\end{defn}

\begin{defn}[Transitivität]
Eine Relation $R \subseteq A^2$ heißt \emph{transitiv} gdw falls $Rxy$
und $Ryz$, dann auch~$Rxz$.
\end{defn}

\begin{defn}[Symmetrie]
Eine Beziehung~$R \subseteq A^2$ heißt \emph{symmetrisch} gdw falls
$Rxy$, dann auch~$Ryx$.
\end{defn}

\begin{defn}[Antisymmetrie]
Eine Relation~$R \subseteq A^2$ ist \emph{anti-sym\-metrisch} gdw falls sowohl
$Rxy$ als auch $Ryx$, dann $x=y$ (oder, anders ausgedrückt: falls $x\neq y$, dann
entweder $\lnot Rxy$ oder $\lnot Ryx$).
\end{defn}

\begin{explain}
In einer symmetrischen Beziehung gelten $Rxy$ und $Ryx$ immer zugleich, oder
gar nicht.  In einer antisymmetrischen Beziehung können $Rxy$
und $Ryx$ nur dann zugleich gelten, wenn $x = y$. Das bedeutet nicht,
dass $Rxy$ und $Ryx$ zugleich gelten \emph{müssen}, wenn $x = y$; nur, dass es
nicht ausgeschlossen ist.  Eine antisymmetrische Beziehung kann also reflexiv sein, aber
es ist nicht so, dass jede antisymmetrische Beziehung reflexiv ist.
Auch ist wichtig, dass \glqq antisymmetrisch\grqq{} nicht dasselbe wie
\glqq nicht symmetrisch\grqq{} ist.  In der Tat kann eine Beziehung sowohl
symmetrisch und antisymmetrisch sein (z. B. die Identitätsbeziehung).
\end{explain}

\begin{defn}[Totalität]
Eine Relation $R \subseteq A^2$ heißt \emph{total} gdw für alle $x,y\in
A$, falls $x \neq y$, dann entweder~$Rxy$ oder~$Ryx$.
\end{defn}

\begin{prob}
Nennen Sie Beispiele für Beziehungen, die (a) reflexiv und symmetrisch, aber
nicht transitiv, (b) reflexiv und antisymmetrisch, (c) antisymmetrisch,
transitiv, aber nicht reflexiv, und (d) reflexiv, symmetrisch und
transitiv sind.  Verwenden Sie keine Beziehungen über Zahlen oder Mengen.
\end{prob} 
 
\begin{defn}[Irreflexivität]
Eine Relation $R \subseteq A^2$ heißt \emph{irreflexiv} gdw für alle $x \in
A$ nicht~$Rxx$ gilt. 
\end{defn}

\begin{defn}[Asymmetrie]
Eine Beziehung $R \subseteq A^2$ heißt \emph{asymmetrisch} gdw für kein Paar $x,y\in
A$ sowohl $Rxy$ als auch~$Ryx$ existieren. 
\end{defn}

Man beachte, dass für $A \neq \emptyset$ keine irreflexive Relation auf~$A$
reflexiv ist und jede asymmetrische Relation auf~$A$ auch
antisymmetrisch ist. Es gibt jedoch $R \subseteq A^2$, die weder
reflexiv noch irreflexiv sind, und es gibt antisymmetrische
Relationen, die nicht asymmetrisch sind. 

\end{document}