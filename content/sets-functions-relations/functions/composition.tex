% Teil: Mengen-Funktionen-Beziehungen
% Kapitel: Funktionen
% Abschnitt: Zusammensetzung

\documentclass[../../../include/open-logic-section]{subfiles}

\begin{document}

\olfileid{sfr}{fun}{cmp}
\olsection{Zusammensetzungen von Funktionen}

\begin{explain}
\oliflabeldef{sfr:fun:inv:sec}{Wir haben in \olref[inv]{sec} gesehen, dass die
Inverse~$f^{-1}$ einer Bijektion~$f$ selbst eine Funktion ist. Eine weitere
Operation auf Funktionen ist die Komposition: w}{W}wir können eine neue Funktion definieren, indem wir zwei Funktionen, $f$ und~$g$,
hintereinander ausführen, d.h.\ indem wir zuerst
$f$ und dann~$g$ anwenden. Das ist natürlich nur möglich, wenn die
Definitions- und Wertebereiche übereinstimmen, d.h.\ der Wertebereich von~$f$ muss eine Teilmenge von
des Definitionsbereichs von~$g$ sein. \oliflabeldef{sfr:rel:ops:sec}{Diese Operation auf
Funktionen ist das Analogon der Operation des relativen Produkts auf
Relationen aus \olref[rel][ops]{sec}.}{}

Ein Diagramm kann helfen, die Idee der Komposition zu erklären. In
\olref{fig:composition} bilden wir zwei Funktionen $f \colon A \to B$
und $g \colon B \to C$ und ihre Komposition~$(\comp{f}{g})$ ab. Die
Funktion $(\comp{f}{g}) \colon A \to C$ paart jedes Element von~$A$
mit einem Element aus~$C$. Wir geben an, welches Element von~$C$
mit Element von $A$ gepaart ist, und zwar foldendermaßen: Bei einem Input $x \in
A$, wende zunächst die Funktion $f$ auf~$x$ an, und erhalte den Output $f(x)
= y \in B$, dann wende die Funktion $g$ auf~$y$ an, die ein
$g(f(x)) = g(y) = z \in C$ ausgibt.
\begin{figure}
  \olasset[2\olphotowidth]{assets/diagrams/composition.tikz}
  \caption{Die Komposition $g \circ f$ von zwei Funktionen $f$ und~$g$.}
  \ollabel{fig:composition}
\end{figure}
\end{explain}

\begin{defn}[Komposition]
Seien $f\colon A \to B$ und $g\colon B \to C$ Funktionen. Die
\emph{Komposition} von $f$ mit~$g$ ist $\comp{f}{g} \colon A \to C$,
wobei $(\comp{f}{g})(x) = g(f(x))$.
\end{defn}

\begin{ex}
Man betrachte die Funktionen $f(x) = x + 1$ und $g(x) = 2x$. Da
$(\comp{f}{g})(x) = g(f(x))$, muss man für jeden Input~$x$ zunächst 
seinen Nachfolger nehmen und dann das Ergebnis mit 2 multiplizieren. Ihre Komposition
ist also gegeben durch $(\comp{f}{g})(x) = 2(x+1)$.
\end{ex}

\begin{prob}
Zeigen Sie, dass wenn $f \colon A \to B$ und $g \colon B \to C$ beide
injektiv sind, dann $\comp{f}{g}\colon A \to C$ injektiv ist.
\end{prob}

\begin{prob}
Zeigen Sie, dass wenn $f \colon A \to B$ und $g \colon B \to C$ beide
surjektiv sind, dann $\comp{f}{g}\colon A \to C$ surjektiv ist.
\end{prob}

\begin{prob}
Angenommen $f \colon A \to B$ und $g \colon B \to C$. Zeigen Sie, dass der Graph
von $\comp{f}{g}$ genau $R_f \mid R_g$ ist.
\end{prob}

\end{document}
