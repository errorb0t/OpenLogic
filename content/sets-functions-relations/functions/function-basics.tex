% Part: sets-functions-relations
% Chapter: functions
% Section: basics

\documentclass[../../../include/open-logic-section]{subfiles}

\begin{document}

\olfileid{sfr}{fun}{bas}
\olsection{Grundlagen}

\begin{explain}
  Eine \emph{Funktion} ist eine Abbildung, die jedes Element einer gegebenen Menge
  zu einem bestimmten Element in einer (anderen) gegebenen Menge schickt. Zum Beispiel definiert die
  Operation des $1$-dazu-Addierens eine Funktion: jede Zahl~$n$ wird abgebildet
  auf eine eindeutige Zahl~$n+1$. 
    
  Allgemeiner ausgedrückt können Funktionen Paare, Tripel usw. als Input annehmen und
 eine Art von Output liefern. Viele Funktionen sind uns aus der elementaren
  Arithmetik vertraut. Zum Beispiel sind die Addition und die Multiplikation
  Funktionen. Sie nehmen zwei Zahlen auf und geben eine dritte zurück.
  
  In diesem mathematischen, abstrakten Sinn ist eine Funktion eine \emph{Black
  Box}: Es kommt nur darauf an, welcher Output mit welchem Input gepaart wird, nicht auf
  die Methode zur Berechnung des Outputs.
  \end{explain}
  
  \begin{defn}[Funktion]
  Eine \emph{Funktion} $f \colon A \to B$ ist eine Abbildung jedes Elements
  aus~$A$ auf ein Element aus~$B$.
  
  Wir nennen $A$ die \emph{Definitionsmenge} von~$f$ und $B$ die \emph{Wertemenge}
  von~$f$.  Die Elemente von~$A$ heißen Inputs oder \emph{Argumente}
  von~$f$, und das Element von~$B$, das mit einem Argument~$x$ gepaart ist
  wird der \emph{(Funktions-)Wert von~$f$} für das Argument~$x$ genannt,
  geschrieben~$f(x)$.
  
  Das \emph{Bild} $\ran{f}$ von~$f$ ist die Teilmenge der Wertemenge
  bestehend aus den Werten von~$f$ für beliebige Argumente; $\ran{f} =
  \Setabs{f(x)}{x \in A}$.
  \end{defn}
  
  Das Diagramm in \olref{fig:function} kann helfen, sich Funktionen vorzustellen. Die Ellipse
  auf der linken Seite stellt die Definitionsmenge der Funktion dar, die Ellipse auf der
  die Ellipse auf der rechten Seite stellt die \emph{Wertemenge} der Funktion dar; und ein Pfeil
  zeigt von einem \emph{Argument} in der Definitionsmenge zum entsprechenden
  \emph{Wert} in der Wertemenge.
  
  \begin{figure}
    \olasset{assets/diagrams/function.tikz}
    \caption{Eine Funktion ist eine Abbildung von jedem Element einer Menge auf
      ein Element einer anderen. Ein Pfeil zeigt von einem Argument im
      Definitionsbereich auf den entsprechenden Wert im Wertebereich.}
    \ollabel{fig:function}
  \end{figure}
  
  \begin{ex}
  Die Multiplikation nimmt Paare von natürlichen Zahlen als Inputs und bildet sie
  auf natürliche Zahlen als Outputs, geht also von $\Nat \times \Nat$ (der Definitionsmenge) zu $\Nat$ (der Wertemenge). Wie sich herausstellt, ist das Bild auch
  $\Nat$, da jedes $n \in \Nat$ $n \times 1$ ist.
  \end{ex}
  
  \begin{ex}
  Die Multiplikation ist eine Funktion, weil sie jede Eingabe - also jedes Paar
  von natürlichen Zahlen - mit einer einzigen Ausgabe paart: $\times \colon \Nat^2 \to
  \Nat$. Im Gegensatz dazu ist die Quadratwurzel auf der Definitionsmenge
  $\Nat$ keine Funktion, da jede positive ganze Zahl $n$ zwei
  Quadratwurzeln hat: $\sqrt{n}$ und $-\sqrt{n}$. Wir können sie zu einer Funktion machen, indem wir
  nur die positive Quadratwurzel ausgeben: $\sqrt{\phantom{X}} \colon
  \Nat \to \Real$. 
  \end{ex}
  
  \begin{ex}
  Die Relation, die jeden Schüler eines Klasse mit seiner Mathenote verbindet,
  ist eine Funktion - kein Schüler kann zwei verschiedene Mathenoten in derselben Klasse erhalten. Die Relation, die jeden Schüler in einer Klasse mit seinen
  Eltern paart, ist keine Funktion: Schüler können null, zwei oder mehr Eltern haben.
  \end{ex}
  
  \begin{explain}
  Wir können Funktionen definieren, indem wir genau angeben, wie der
  Wert der Funktion für jedes mögliche Argment ist. Verschiedene Möglichkeiten
  sind die Angabe einer Formel, die Beschreibung einer Methode zur Berechnung
  des Wertes oder die Auflistung der Werte für jedes Argument. Wie auch immer Funktionen
  definiert werden, müssen wir sicherstellen, dass wir für jedes Argument einen,
  und nur einen, Wert angeben.
  \end{explain}
  
  
  \begin{ex}
  Sei $f \colon \Nat \to \Nat$ so definiert, dass $f(x) = x+1$. Das
  ist eine Definition, die $f$ als eine Funktion spezifiziert, die
  natürliche Zahlen aufnimmt und natürliche Zahlen ausgibt. Sie sagt uns, dass $f$ für eine
  natürliche Zahl~$x$ ihren Nachfolger~$x+1$ ausgibt.
  In diesem Fall ist die Wertemenge $\Nat$ nicht das Bild von~$f$, da die
  natürliche Zahl~$0$ nicht der Nachfolger einer natürlichen Zahl ist. Der
  Bereich von~$f$ ist die Menge aller positiven ganzen Zahlen, $\Int^{+}$.
  \end{ex}
  
  \begin{ex}\ollabel{examplefunext}
  Es sei $g \colon \Nat \to \Nat$ so definiert, dass $g(x) = x+2-1$. Diese
  sagt uns, dass $g$ eine Funktion ist, die natürliche Zahlen aufnimmt und
  natürliche Zahlen ausgibt. Für eine natürliche Zahl~$n$ gibt $g$
  den Vorgänger des Nachfolgers des Nachfolgers von~$x$, aus d.h.
  $x+1$.
  \end{ex}
  
  \begin{explain}
  Wir haben gerade zwei Funktionen, $f$ und $g$, mit unterschiedlichen
  \emph{Definitionen}. Es handelt sich jedoch um dieselbe Funktion.
  Schließlich gilt für jede natürliche Zahl~$n$, dass $f(n) = n+1 = n+2-1 =
  g(n)$. Anders ausgedrückt: Unsere Definitionen für $f$ und~$g$ geben
  dieselbe Abbildung mit Hilfe verschiedener Gleichungen an. Implizit stützen wir uns also
  auf ein Prinzip der Extensionalität für Funktionen, 
  \[
    \text{falls }\forall x\, f(x) = g(x)\text{, dann }f = g
  \]
  vorausgesetzt, dass $f$ und~$g$ denselben Definitionsbereich und denselben Wertebereich haben.
  \end{explain}
  
  \begin{ex}
  Wir können Funktionen auch durch Fälle definieren. Zum Beispiel könnten wir definieren
  $h \colon \Nat \to \Nat$ durch
  \[
  h(x) =
  \begin{cases}
    \frac{x}{2} & \text{falls $x$ gerade ist} \\
    \frac{x+1}{2} & \text{falls $x$ ungerade ist.}
  \end{cases}
  \]

  Da jede natürliche Zahl entweder gerade oder ungerade ist, ist die Ausgabe dieser
  Funktion immer eine natürliche Zahl. Wann immer wir
  eine Funktion durch Fälle definieren, muss jede mögliche Eingabe
  genau einem Fall entsprechen.  In manchen Fällen erfordert dies einen Beweis, dass die
  Fälle vollständig und einander ausschließend vorliegen.
  \end{ex}
  
  \end{document}