% Teil: Mengen-Funktionen-Beziehungen
% Kapitel: Funktionen
% Abschnitt: kinds

\documentclass[../../../include/open-logic-section]{subfiles}

\begin{document}

\olfileid{sfr}{fun}{kin}
\olsection{Arten von Funktionen}

\begin{explain}
Es wird nützlich sein, eine Art Klassifizierung für einige der
Arten von Funktionen einzuführen, die uns am häufigsten begegnen. 

Für den Anfang könnten wir Funktionen betrachten, die die Eigenschaft haben,
dass jedes Element der Wertemenge ein Wert der Funktion ist. Solche
Funktionen werden surjektiv genannt und können wie folgt dargestellt werden
\olref{fig:surjective}.

\begin{figure}
  \olasset{assets/diagrams/surjective.tikz}
  \caption{eine surjektive Funktion bildet auf jedes Element der
    Wertemenge ab.}
  \ollabel{fig:surjective}
\end{figure}
\end{explain}

\begin{defn}[surjektive Funktion]
Eine Funktion $f \colon {A \rightarrow B}$ heißt \emph{surjektiv} gdw $B$
auch die Wertemenge von~$f$ ist, d.h. für jedes $y \in B$ gibt es mindestens
ein $x \in A$, sodass~$f(x) = y$, oder in Zeichen:
\[
  (\forall y \in B)(\exists x \in A)f(x) = y.
\]
Wir nennen eine solche Funktion Surjektion von $A$ nach $B$.
\end{defn}

\begin{explain}
Wenn man zeigen will, dass $f$ eine Surjektion ist, dann muss man zeigen
dass jedes Objekt in $f$'s Wertemenge der Wert von $f(x)$ für eine
Eingabe $x$ ist.

Man beachte, dass jede Funktion eine Surjektion \emph{induziert}. Denn:
Für eine Funktion $f \colon A \to B$ sei $f' \colon A \to \ran{f}$
definiert durch $f'(x) = f(x)$. Da $\ran{f}$ \emph{definiert} ist als
$\Setabs{f(x) \in B}{x \in A}$, ist diese Funktion $f'$ garantiert
surjektiv.
\end{explain}

\begin{explain}
Nun bildet jede Funktion jeden möglichen Input auf einen eindeutigen Output ab. Aber
es gibt auch Funktionen, die niemals verschiedene Inputs auf denselben Output abbilden.
Solche Funktionen nennt man injektiv, und man kann sie sich vorstellen
wie in \olref{fig:injective}.
\end{explain}

\begin{defn}[injektive Funktion] 
Eine Funktion $f \colon A \rightarrow B$ ist \emph{injektiv} gdw für
jedes $y \in B$ höchstens ein $x \in A$ existiert, sodass~$f(x) = y$. Wir
nennen eine solche Funktion eine Injektion von $A$ nach~$B$.
\end{defn}

\begin{explain}
Wenn man zeigen will, dass $f$ eine Injektion ist, muss man zeigen, dass
für beliebige Elemente $x$ und $y$ der Wertemenge von $f$ gilt: Falls $f(x)=f(y)$, dann
$x=y$. 
\begin{figure}
  \olasset{assets/diagrams/injective.tikz}
  \caption{Eine injektive Funktion bildet niemals zwei unterschiedliche
    Argumente auf denselben Wert ab.}
  \ollabel{fig:injective}
\end{figure}
\end{explain}

\begin{ex}
Die konstante Funktion $f\colon \Nat \to \Nat$, gegeben durch $f(x) = 1$, ist
weder injektiv, noch surjektiv.

Die Identitätsfunktion $f\colon \Nat \to \Nat$, gegeben durch $f(x) = x$, ist
sowohl injektiv als auch surjektiv.

Die Nachfolgefunktion $f \colon \Nat \to \Nat$, gegeben durch $f(x) = x+1$
ist injektiv, aber nicht surjektiv.
  
Die Funktion $f \colon \Nat \to \Nat$, definiert durch:
\[
  f(x) =
  \begin{cases}
    \frac{x}{2} & \text{falls $x$ gerade ist} \\
    \frac{x+1}{2} & \text{falls $x$ ungerade ist.}
  \end{cases}
\]
ist surjektiv, aber nicht injektiv.
\end{ex}

\begin{explain}
Oft wollen wir Funktionen betrachten, die sowohl
injektiv als auch surjektiv sind. Wir nennen solche Funktionen
bijektiv. Sie sehen aus wie die Funktion, die in
\olref{fig:bijective}. Bijektionen werden manchmal auch als
\emph{Eins-zu-Eins-Korrespondenzen} genannt, da sie eindeutig Elemente
des Wertemenge mit Elementen der Definitionsmenge verbinden.
\begin{figure}
  \olasset{assets/diagrams/bijective.tikz}
  \caption{Eine bijektive Funktion paart eindeutig die Elemente der
    Wertemenge mit denen der Definitionsmenge.}
  \ollabel{fig:bijective}
\end{figure}
\end{explain}

\begin{defn}[Bijektion] 
Eine Funktion $f \colon A \to B$ heißt \emph{bijektiv}, wenn sie sowohl
surjektiv als auch injektiv ist. Wir nennen eine solche Funktion
Bijektion von $A$ nach~$B$ (oder zwischen $A$ und~$B$).
\end{defn}

\end{document}
