% Teil: Mengen-Funktionen-Beziehungen
% Kapitel: Funktionen
% Abschnitt: partial-functions

\documentclass[../../../include/open-logic-section]{subfiles}


\begin{document}

\olfileid{sfr}{fun}{par}

\olsection{Partielle Funktionen}

\begin{explain}
Manchmal ist es sinnvoll, die Definition einer Funktion so zu lockern,
dass die Ausgabe der Funktion nicht für \emph{alle} möglichen
möglichen Eingaben definiert sein muss. Solche Zuordnungen werden \emph{partielle Funktionen} genannt.
\end{explain}

\begin{defn}
Eine \emph{partielle Funktion} $f \colon A \pto B$ ist eine Abbildung, die
jedem Element von~$A$ höchstens ein Element von~$B$ zuordnet.
Wenn $f$ ein Element von~$B$ zu $x \in A$ zuordnet, nennt man $f(x)$
\emph{definiert}, und andernfalls \emph{undefiniert}. Wenn $f(x)$ definiert ist,
schreiben wir $f(x) \fdefined$, andernfalls $f(x) \fundefined$. Der
\emph{Definitionsbereich} einer Teilfunktion~$f$ ist die Teilmenge von~$A$, in der sie
definiert ist, d.h. $\dom{f} = \Setabs{x \in A}{f(x) \fdefined}$.
\end{defn}

\begin{ex}
Jede Funktion $f \colon A \to B$ ist auch eine partielle Funktion. Partielle
Funktionen, die überall auf~$A$ definiert sind---also das, was wir bisher
einfach eine Funktion genannt haben---werden auch \emph{total}
genannt.
\end{ex}

\begin{ex}
Die Teilfunktion $f \colon \Real \pto \Real$, gegeben durch $f(x) = 1/x$,
ist für $x = 0$ undefiniert, und sonst überall definiert.
\end{ex}

\begin{prob}
Definiere für $f\colon A \pto B$ die Teilfunktion $g\colon B \pto
A$ durch folgende Vorschrift: für jedes $y \in B$, falls es ein eindeutiges $x \in A$ gibt,
so dass $f(x) = y$, setze $g(y) = x$; andernfalls $g(y) \fundefined$.  Zeigen Sie, dass
wenn $f$ injektiv ist, dann $g(f(x)) = x$ für alle $x \in \dom{f}$, und
$f(g(y)) = y$ für alle $y \in \ran{f}$.
\end{prob}

\begin{defn}[Graph einer partiellen Funktion]
Sei $f\colon A \pto B$ eine Teilfunktion. Der \emph{Graph} von~$f$
ist die Relation $R_f \subseteq A \times B$, definiert durch
\[
R_f = \Setabs{\tuple{x,y}}{f(x) = y}.
\]
\end{defn}

\begin{prop}
Angenommen, $R \subseteq A \times B$ hat die Eigenschaft, dass immer dann, wenn $Rxy$
und $Rxy'$ dann $y = y'$.  Dann ist $R$ der Graph der partiellen
Funktion $f\colon X \pto Y$, definiert durch: Wenn es ein $y$ gibt, das so ist, dass
$Rxy$, dann ist $f(x) = y$, andernfalls $f(x) \fundefined$.  Wenn für $R$ auch gilt,
dass es für jedes $x \in X$ ein $y \in Y$ gibt, so dass
$Rxy$, dann ist $f$ total.
\end{prop}

\begin{proof}
Angenommen, es gibt ein $y$, für das $Rxy$ gilt.  Wenn es ein anderes $y' gäbe
\neq y$, so dass $Rxy'$, wäre die Bedingung für $R$
verletzt. Wenn es also ein $y$ gibt, für das $Rxy$ gilt, dann ist dieses $y$
eindeutig, und somit ist $f$ wohldefiniert.  Offensichtlich ist $R_f = R$ und $f$ ist
total, wenn es für jedes $x \in X$ ein $y \in Y$ gibt, so dass
$Rxy$ ist.
\end{proof}

\end{document}