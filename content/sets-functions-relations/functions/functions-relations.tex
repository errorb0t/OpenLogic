% Teil: Mengen-Funktionen-Beziehungen
% Kapitel: Funktionen
% Abschnitt: functions-relations

\documentclass[../../../include/open-logic-section]{subfiles}


\begin{document}

\olfileid{sfr}{fun}{rel}

\olsection{Funktionen als Relationen}

\begin{explain} 
Eine Funktion, die Elemente aus~$A$ auf Elemente aus~$B$ abbildet,
definiert offensichtlich eine Relation zwischen $A$ und~$B$, nämlich die Relation,
die zwischen $x$ und $y$ besteht, falls $f(x) = y$.  In der Tat, wir könnten
sogar---wenn wir daran interessiert sind, die Grundbausteine der
Mathematik zu reduzieren---die Funktion~$f$ mit dieser Relation \emph{identifizieren},
d.h. mit einer Menge von Paaren. Das wirft dann die Frage auf:
Welche Relationen definieren Funktionen auf diese Weise?
\end{explain}

\begin{defn}[Graph einer Funktion] Sei $f\colon A \to B$ eine Funktion.
Der \emph{Graph} von~$f$ ist die Relation $R_f \subseteq A \times B$
definiert durch
\[
R_f = \Setabs{\tuple{x,y}}{f(x) = y}.
\]
\end{defn}

\begin{explain}
Der Graph einer Funktion ist durch die Extensionalität eindeutig bestimmt.
Außerdem bestätigt die Extensionalität (auf Mengen) unmittelbar das
implizite Prinzip der Extensionalität für Funktionen,
wonach, falls $f$ und~$g$ sich Definitions- und Wertemenge teilen, identisch sind,
wenn sie in allen Funktionswerten übereinstimmen. 

Wenn eine Beziehung \glqq funktional\grqq{} ist, dann ist sie der Graph einer Funktion. 
\end{explain}

\begin{prop}\ollabel{prop:graph-function}
Sei $R \subseteq A \times B$ sodass:
\begin{enumerate}
\item Falls $Rxy$ und $Rxz$ dann $y = z$; und 
\item für jedes $x \in A$ gibt es ein $y \in B$ mit $\tuple{x,
y} \in R$.  
\end{enumerate}
Dann ist $R$ der Graph der Funktion $f\colon A \to B$ definiert durch
$f(x) = y$ gdw $Rxy$. 
\end{prop}

\begin{proof}
Angenommen, es gibt ein $y$, für das $Rxy$ gilt.  Gäbe es ein anderes $z \neq
y$, so dass $Rxz$, würde die Bedingung, die wir an~$R$ gestellt haben, verletzt werden. Wenn es
also ein $y$ gibt, für das $Rxy$ gilt, so ist dieses $y$ eindeutig, und $f$ ist somit
wohldefiniert.  Es ist offensichtlich, dass $R_f = R$ ist.
\end{proof}

\begin{explain}
Jede Funktion $f\colon A \to B$ hat einen Graphen, d.h. eine Beziehung auf $A
\times B$, definiert durch $f(x) = y$. Andererseits ist jede Relation~$R
\subseteq A \times B$ mit den Eigenschaften aus
\olref{prop:graph-function} der Graph einer Funktion~$f \colon A \to
B$. Wegen dieser engen Verbindung zwischen Funktionen und ihren
Graphen können wir uns eine Funktion einfach als ihren Graphen vorstellen.
Mit anderen Worten: Funktionen können mit bestimmten Relationen identifiziert werden, d.h. mit
bestimmten Mengen von Tupeln. \oliflabeldef{sfr:rel:ref:sec}{Man beachte jedoch,
dass der Sinn dieser ``Identifikation'' der gleiche ist wie in
\olref[sfr][rel][ref]{sec}: Es handelt sich nicht um eine Behauptung über die Metaphysik von
Funktionen, sondern eine Beobachtung, dass es praktisch ist,
Funktionen als bestimmte Mengen zu \emph{behandeln}. Ein Grund, warum das so praktisch ist folgender:} 
Wir können nun in Betracht ziehen, ähnliche Operationen mit Funktionen durchzuführen, 
ie wir es bei Relationen getan haben (siehe
\olref[sfr][rel][ops]{sec}). Insbesondere:
\end{explain}

\begin{defn}\ollabel{defn:funimage}
Sei $f \colon A \to B$ eine Funktion mit $C\subseteq A$.

Die \emph{Restriktion} von~$f$ auf~$C$ ist die
Funktion~$\funrestrictionto{f}{C}\colon C \to B$ definiert durch
$(\funrestrictionto{f}{C})(x) = f(x)$ für alle $x \in C$.
Mit anderen Worten, $\funrestrictionto{f}{C} = \Setabs{\tuple{x, y} \in R_f}{x \in
C}$.

Das \emph{Bild} von~$C$ unter~$f$ ist $\funimage{f}{C} =
\Setabs{f(x)}{x \in C}$.
\end{defn}

\begin{explain}
Aus diesen Definitionen folgt für jede Funktion~$f$, dass $\ran{f} =
\funimage{f}{\dom{f}}$.
\oliflabeldef{sfr:rel:ops:sec}{Diese Begriffe sind genau so, wie man sie
erwarten würde, wenn man die Definitionen in \olref[sfr][rel][ops]{sec} und unsere
Identifizierung von Funktionen mit Relationen heranzieht. Aber zwei andere
Operationen - Inverse und relative Produkte - erfordern ein wenig mehr
Feinarbeit. Diese werden wir in den Kapiteln \olref[inv]{sec} und
\olref[cmp]{sec} leisten.}{}
\end{explain}

\end{document}


Übersetzt mit www.DeepL.com/Translator (kostenlose Version)