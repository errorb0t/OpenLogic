% Teil: Mengen-Funktionen-Beziehungen
% Kapitel: Funktionen
% Abschnitt: Inversen

\documentclass[../../../include/open-logic-section]{subfiles}

\begin{document}

\olfileid{sfr}{fun}{inv}
\olsection{Inverse von Funktionen}

\begin{explain}
Wir denken uns Funktionen als Abbildungen. Eine offensichtliche Frage, die man sich bei
Funktionen stellen kann, ist also, ob die Abbildung \glqq umgekehrt\grqq{} werden kann. Unter
zum Beispiel kann die Nachfolgefunktion $f(x) = x + 1$ umgekehrt werden, in dem
in dem Sinne, dass die Funktion $g(y) = y - 1$ das, was $f$ tut, \glqq umkehrt\grqq{}. 

Aber es ist Vorsicht geboten. Obwohl~$g$ eine
Funktion $\Int \to \Int$ definiert, definiert sie keine \emph{Funktion} $\Nat
\to \Nat$, da $g(0) \notin \Nat$ liegt.  Selbst in einfachen Fällen ist es also
nicht ganz klar, ob eine Funktion umgekehrt werden kann; es kann von dem Definitions-
und vom Geltungsbereich abhängen.

Dies wird durch den Begriff der Inversen einer Funktion präzisiert.
\end{explain}

\begin{defn}
Eine Funktion $g \colon B \to A$ ist eine \emph{Inverse} einer Funktion $f
\colon A \to B$, falls $f(g(y)) = y$ und $g(f(x)) = x$ für alle $x \in A$
und $y \in B$.
\end{defn}

Wenn $f$ eine Inverse~$g$ hat, schreibt man oft $f^{-1}$ anstelle von~$g$.

\begin{explain}
Nun wollen wir bestimmen, wann Funktionen Inverse haben. Ein guter Kandidat
für eine Inverse von $f\colon A \to B$ ist $g\colon B \to A$ \glqq definiert\grqq{}
\[
g(y) = \text{\glqq die\grqq{} $x$, sodass $f(x) = y$}
\]
Aber die Anführungszeichen um \glqq definiert\grqq{} durch (und \glqq die\grqq{}) deuten darauf hin, dass
dies keine Definition ist.  Zumindest wird sie nicht in vollständiger Allgemeinheit funktionieren.
Denn damit diese Definition eine Funktion spezifizieren kann, muss es genau ein~$x$ geben,
so dass $f(x) =
y$---der Output von~$g$ muss eindeutig spezifiziert sein. Außerdem muss es
für jedes $y \in B$ spezifiziert sein.  Wenn es $x_1$ und $x_2 \in
A$ mit $x_1 \neq x_2$ aber $f(x_1) = f(x_2)$ gibt, dann wäre $g(y)$ nicht
nicht eindeutig spezifiziert für $y = f(x_1) = f(x_2)$. Und wenn es kein~$x$
gibt, so dass $f(x) = y$, dann ist $g(y)$ überhaupt nicht spezifiziert.  In
Mit anderen Worten: Damit $g$ definiert werden kann, muss $f$~sowohl injektiv als auch
surjektiv sein.

Gehen wir langsam vor. Wir werden die Frage in zwei Teile aufteilen: Für eine
Funktion~$f\colon A \to B$---wann gibt es eine Funktion $g\colon B \to A$
so dass $g(f(x)) = x$ ist? Ein solches $g$ \glqq macht rückgängig\grqq{}, was $f$ tut, und wird
ein \emph{Linksinverse} von~$f$ genannt. Zweitens, wann gibt es eine
Funktion $h \colon B \to A$, so dass $f(h(y)) = y$? Ein solches $h$ wird
ein \emph{Rechtsinverse} von~$f$ genannt---$f$ \glqq macht rückgängig\grqq{}, was $h$~tut.
\end{explain}

\begin{prop}
Wenn $f\colon A \to B$ injektiv ist, dann gibt es eine 
\emph{Linksinverse}~$g\colon B \to A$ von~$f$, so dass $g(f(x)) = x$ für alle $x
\in A$.
\end{prop}

\begin{proof}
Nehmen wir an, dass $f\colon A \to B$ injektiv ist. Betrachten Sie ein $y \in B$.
Wenn $y \in \ran{f}$, gibt es ein $x \in A$, so dass $f(x) = y$. Weil
$f$ injektiv ist, gibt es nur ein solches~$x \in A$. Dann können wir
definieren: $g(y) = x$, d.h. $g(y)$ ist \emph{das} $x \in A$, so dass $f(x)
= y$.  Wenn $y \notin \ran{f}$ ist, können wir es auf jedes~$a \in A$ abbilden. Wir können also
ein $a \in A$ wählen und $g \colon B \to A$ definieren durch:
\[
g(y) = \begin{cases}
    x & \text{falls $f(x) = y$}\\
    a & \text{falls $y \notin \ran{f}$.}
\end{cases}
\]
$g$ ist für alle $y \in B$ definiert, da für jedes solche $y \in \ran{f}$
genau ein $x \in A$ gibt, für das $f(x) = y$ gilt. Per Definition, wenn
$y = f(x)$, dann ist $g(y) = x$, d.h. $g(f(x)) = x$.
\end{proof}

\begin{prob}
Zeigen Sie, dass, wenn $f\colon A \to B$ eine linke Inverse~$g$ hat, dann ist $f$~injektiv.
\end{prob}

\begin{prop}
    Wenn $f \colon A \to B$ surjektiv ist, dann gibt es eine
    \emph{Rechtsinverse}~$h\colon B \to A$ von~$f$, so dass $f(h(y)) =
    y$ für alle~$y \in B$.
\end{prop}

\begin{proof}
Nehmen wir an, dass $f\colon A \to B$ surjektiv ist. Betrachten Sie ein $y \in
B$. Da $f$~surjektiv ist, gibt es ein $x_y \in A$ mit $f(x_y)
= y$.  Dann können wir definieren: $h(y) = x_y$, d.h.\ für jedes $y \in B$ wählen wir
irgendein $x \in A$, so dass $f(x) = y$; da $f$~surjektiv ist,
gibt es immer mindestens eine zur Auswahl.\footnote{Da $f$
surjektiv ist, ist für jedes~$y \in B$ die Menge $\Setabs{x}{f(x) = y}$
nicht-leer. Unsere Definition von~$h$ erfordert, dass wir ein einziges $x$
aus jeder dieser Mengen wählen. Dass dies immer möglich ist, ist eigentlich nicht
offensichtlich---die Möglichkeit, diese Auswahl zu treffen, wird einfach als ein Axiom vorausgesetzt. 
Mit anderen Worten, dieser Satz setzt das so genannte Auswahlaxiom voraus, ein Thema,
\oliflabeldef{sth:choice::chap}{auf das wir in zurückkommen werden in \olref[sth][choice][]{chap}}{das wir hier nicht weiter behandeln}. 
In vielen Fällen, z.B. wenn $A = \Nat$ oder endlich ist, oder falls $f$ bijektiv ist,
ist das Auswahlaxiom nicht erforderlich. (In dem besonderen Fall, dass $f$ bijektiv ist,
ist für jedes $y \in B$ die Menge 
$\Setabs{x}{f(x) = y}$ genau ein Element, sodass keine Wahl zu treffen ist.)} 
Nach Definition, falls $x = h(y)$,
dann $f(x) = y$, d.h.\ für jedes $y \in B$ ist $f(h(y)) = y$.
\end{proof}

\begin{prob}
Zeigen Sie, dass, wenn $f\colon A \to B$ eine rechte Inverse~$h$ hat, $f$~surjektiv ist.
\end{prob}

\begin{explain}
  Indem wir die Ideen des vorherigen Beweises kombinieren, erhalten wir nun, dass jede
  Bijektion eine Inverse hat, d.h.\ es gibt eine einzige Funktion,
  die sowohl eine linke als auch eine rechte Inverse von~$f$ ist.
\end{explain}

\begin{prop}\ollabel{prop:bijection-inverse}
Wenn $f\colon A \to B$ bijektiv ist, gibt es eine
Funktion~$f^{-1}\colon B \to A$, so dass für alle $x \in A$,
$f^{-1}(f(x)) = x$ und für alle $y \in B$, $f(f^{-1}(y)) = y$.
\end{prop}

\begin{proof}
Übung.
\end{proof}

\begin{prob}
Beweisen Sie \olref[sfr][fun][inv]{prop:bijection-inverse}. Man muss~$f^{-1}$ definieren,
zeigen, dass es sich um eine Funktion handelt, und zeigen, dass es eine
Inverse von~$f$ ist, d.h. $f^{-1}(f(x)) = x$ und $f(f^{-1}(y)) = y$ für
alle $x \in A$ und $y \in B$.
\end{prob}

\begin{explain}
Es gibt einen etwas allgemeineren Weg, Inverse zu berechnen. Wir haben in
\olref[kin]{sec} gesehen, dass jede Funktion $f$ eine Surjektion $f'
\colon A \to \ran{f}$ induziert, indem man $f'(x) = f(x)$ für alle $x \in A$ setzt.
Es ist klar, wenn $f$~injektiv ist, dann ist $f'$~bijektiv, so dass
$f$ eine eindeutige Inverse nach \olref{prop:bijection-inverse} hat.
Als Notationsbagatelle nennen wir die Inverse von $f'$ manchmal einfach
\glqq die Inverse von~$f$.\grqq{}
\end{explain}

\begin{prop}\ollabel{prop:left-right}%
  Zeigen Sie, dass, wenn $f\colon A \to B$ eine Linksinverse~$g$ und eine Rechtsinverse~$h$ hat,
  dann ist $h = g$.
\end{prop}

\begin{proof}
  Übung.
\end{proof}

\begin{prob}
  Beweisen Sie \olref[sfr][fun][inv]{prop:left-right}.
\end{prob}

\begin{prop}\ollabel{prop:inverse-unique}
Jede Funktion~$f$ hat höchstens eine Inverse.
\end{prop}

\begin{proof}
  Angenommen, $g$ und $h$ sind beide Inverse von~$f$. Dann ist insbesondere
  $g$~eine Linksinverse von~$f$ und $h$~eine Rechtsinverse. Nach
  \olref{prop:left-right} gilt $g = h$.
\end{proof}

\end{document}

