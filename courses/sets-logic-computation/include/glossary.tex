\newglossaryentry{dtheorem}{
  name = {deduction theorem},
  description = {Relates entailment and provability of a sentence from
    an assumption with that of a corresponding conditional. In the
    semantic form (\olref[fol][syn][sem]{thm:sem-deduction}), it states
    that $\Gamma \cup \{!A\} \Entails !B$ iff $\Gamma \Entails !A \lif
    !B$. In the proof-theoretic form, it states that $\Gamma \cup
   \{!A\} \Proves !B$ iff $\Gamma \Proves !A \lif !B$}}

\newglossaryentry{set-extensionality} {
  name={extensionality (of sets)},  
  description={Sets $A$ and~$B$ are identical, $A = B$, iff every
    element of~$A$ is also an element of~$B$, and vice versa
    (see \olref[sfr][set][bas]{sec})}}

\newglossaryentry{sat-extensionality}{
        name = {extensionality (of satisfaction)},
        description = {Whether or not a formula~$!A$ is satisfied
        depends only on the assignments to the non-logical symbols and
        free variables that actually occur in~$!A$}}

\newglossaryentry{set} {
  name=set,  
  description={A collection of objects, considered independently of
    the way it is specified, of the order of the objects in the set,
    and of their multiplicity (see \olref[sfr][set][bas]{sec})} }

\newglossaryentry{symmetric} {
  name=symmetric,  
  description={$R$ is symmetric iff, whenever $Rxy$ then also $Ryx$
    (see \olref[sfr][rel][prp]{sec})} }

\newglossaryentry{cp} {
  name=Cartesian product,
  symbol = {\ensuremath{A \times B}},  
  description={Set of all pairs of {elements} of $A$ and $B$; $A
    \times B = \Setabs{\tuple{x, y}}{x \in A \text{ and } y \in B}$
    (see \olref[sfr][set][pai]{sec})} }

\newglossaryentry{subset} {
  name=subset,
  symbol = {\ensuremath{A \subseteq B}},
  description = {A set every {element} of which is an {element} of
    a given set~$B$ (see \olref[sfr][set][sub]{sec})}  }

\newglossaryentry{finsequence} {
  name = {sequence (finite)},
  symbol = {\ensuremath{A^*}},
  description = {A finite string of {element}s of~$A$; an
    {element} of $A^n$ for some $n$ (see
    \olref[sfr][set][imp]{sec})}}

\newglossaryentry{infsequence} {
  name = {sequence (infinite) },
  symbol = {\ensuremath{A^\omega}},
  description = {A gapless, unending sequence of {element}s of~$A$;
    formally, a function $s\colon \PosInt \to A$ (see
    \olref[sfr][set][imp]{sec})}}

\newglossaryentry{ps} {
  name=power set,
  symbol = {\ensuremath{\Pow{A}}},  
  description={The set consisting of all \glspl{subset} of a set~$A$,
    $\Pow{A} = \Setabs{x}{x \subseteq A}$ (see
    \olref[sfr][set][sub]{sec})} }

\newglossaryentry{union} {
  name=union,
  symbol = {\ensuremath{A \cup B}},
  description={The set of all {element}s of $A$ and $B$ together: $A
    \cup B = \Setabs{x}{x \in A \lor x\in B}$ (see
    \olref[sfr][set][uni]{sec})} }

\newglossaryentry{intersection} {
  name=intersection,
  symbol = {\ensuremath{A \cap B}},
  description={The set of all things which are {element}s of both
    $A$ and~$B$: $A \cap B = \Setabs{x}{x \in A \land x\in B}$ (see
    \olref[sfr][set][uni]{sec})} }

\newglossaryentry{difference} {
  name=difference,
  symbol = {\ensuremath{A \setminus B}},
  description={the set of all {elements} of $A$ which are not also
  {element}s of~$B$: $A \setminus B = \Setabs{x}{x \in A \text{ and }
    x \notin B}$ (see \olref[sfr][set][uni]{sec})} }

\newglossaryentry{disjoint}{
  name=disjoint, 
  description={two sets with no {element}s in common (see
    \olref[sfr][set][uni]{sec})}}

\newglossaryentry{composition} {
  name=composition,
  symbol = {\ensuremath{g \circ f}},  
  description={The function resulting from ``chaining together'' $f$
    and $g$; $(g \circ f)(x) = g(f(x))$ (see
    \olref[sfr][fun][cmp]{sec})} }

\newglossaryentry{function} {
  name=function,
  symbol = {\ensuremath{f \colon A \to B}},
  description={A mapping of each element of a \gls{fdomain}~$A$ to an
    element of the codomain~$B$ (see \olref[sfr][fun][bas]{sec})} }

\newglossaryentry{fdomain} {
  name = {domain (of a function)},
  symbol = {\ensuremath{\dom{f}}},
  description = {The set of objects for which a (partial) function is
    defined (see \olref[sfr][fun][bas]{sec})}}

\newglossaryentry{range}{
  name = range,
  symbol = {\ensuremath{\ran{f}}},
  description = {the subset of the codomain that is actually output by
    $f$; $\ran{f} = \Setabs{y \in B}{f(x) = y \text{ for some $x \in
        A$}}$ (see \olref[sfr][fun][bas]{sec})}}

\newglossaryentry{graph} {
  name=graph (of a function),
  description={the relation $R_f \subseteq A \times B$ defined by $R_f
    = \Setabs{\tuple{x,y}}{f(x) = y}$, if $f\colon A \pto B$ (see
    \olref[sfr][fun][rel]{sec})} }

\newglossaryentry{inverse} {
  name=inverse function,
  description={If $f\colon A \to B$ is a \gls{bijection}, $f^{-1} \colon
    B \to A$ is the function with $f^-1(y) =$ whatever unique $x \in
    A$ is such that $f(x) = y$ (see \olref[sfr][fun][inv]{sec})} }

\newglossaryentry{pf} {
  name=partial function,
    symbol = {\ensuremath{f \colon A \pto B}},
  description={A partial function is a mapping which assigns to every
    {element} of~$A$ at most one {element} of~$B$. If $f$ assigns
    an element of~$B$ to $x \in A$, $f(x)$ is defined, and otherwise
    undefined (see \olref[sfr][fun][par]{sec})} }

\newglossaryentry{tc} {
  name=transitive closure,
  symbol = {\ensuremath{R^+}},
  description={the smallest \gls{transitive} relation containing~$R$
    (see \olref[sfr][rel][ops]{sec})} }

\newglossaryentry{ir} {
  name=inverse relation,
  symbol = {\ensuremath{R^{-1}}},  
  description={The relation $R$ ``turned around''; $R^{-1} =
    \Setabs{\tuple{y, x}}{\tuple{x, y} \in R}$ (see
    \olref[sfr][rel][ops]{sec})} }

\newglossaryentry{preorder} {
  name=preorder,
  description={A \gls{reflexive} and \gls{transitive} relation (see
    \olref[sfr][rel][ord]{sec})} }

\newglossaryentry{po} {
  name=partial order,
  description={A \gls{reflexive}, \gls{anti-symmetric},
    \gls{transitive} relation (see \olref[sfr][rel][ord]{sec})} }

\newglossaryentry{to} {
  name=total order,
  description={see \gls{lo}}}

\newglossaryentry{irreflexive} {
  name=irreflexive,
  description={$R$ is irreflexive if, for no $x \in A$, $Rxx$ (see
    \olref[sfr][rel][ord]{sec})} }

\newglossaryentry{asymmetric} {
  name=asymmetric,
  description={$R$ is asymmetric if for no pair $x,y\in A$ we have
    $Rxy$ and $Ryx$ (see \olref[sfr][rel][ord]{sec})} }

\newglossaryentry{so} {
  name=strict order,
  description={An \gls{irreflexive}, \gls{asymmetric}, and
    \gls{transitive} relation (see \olref[sfr][rel][ord]{sec})} }

\newglossaryentry{slo} {
  name=strict linear order,
  description={A connected \gls{so} (see
    \olref[sfr][rel][ord]{sec})} }

\newglossaryentry{lo} {
  name=linear order,
  description={A connected \gls{po} (see
    \olref[sfr][rel][ord]{sec})} }

\newglossaryentry{br} {
  name=binary relation,
  description={A subset of~$A^{2}$; we write $Rxy$ (or $xRy$) for
    $\tuple{x, y} \in R$ (see \olref[sfr][rel][set]{sec})} }

\newglossaryentry{reflexive} {
  name=reflexive,
  description={$R$ is reflexive iff, for every $x \in A$, $Rxx$ (see
    \olref[sfr][rel][prp]{sec})} }

\newglossaryentry{transitive} {
  name=transitive,
  description={$R$ is transitive iff, whenever $Rxy$ and $Ryz$, then
    also $Rxz$ (see \olref[sfr][rel][prp]{sec})} }

\newglossaryentry{anti-symmetric} {
  name=anti-symmetric,
  description={$R$ is anti-symmetric iff, whenever both $Rxy$ and
    $Ryx$, then $x=y$; in other words: if $x\neq y$ then not $Rxy$ or
    not $Ryx$ (see \olref[sfr][rel][prp]{sec})} }

\newglossaryentry{connected} {
  name=connected,
  description={$R$ is connected if for all $x, y\in A$
    with $x \neq y$, then either $Rxy$ or~$Ryx$ (see
    \olref[sfr][rel][prp]{sec})} }

\newglossaryentry{er} {
  name={equivalence relation}, 
  description={a reflexive, symmetric, and transitive relation (see
    \olref[sfr][rel][prp]{sec})} }

\newglossaryentry{sf} {
  name={surjective},
  description={$f \colon A \to B$ is {surjective} iff the
    range of~$f$ is all of~$B$, i.e., for every $y \in B$ there is at
    least one $x \in A$ such that~$f(x) = y$ (see
    \olref[sfr][fun][kin]{sec})} }

\newglossaryentry{if} {
  name={injective}, 
  description={$f \colon A \to B$ is {injective} iff for
    each $y \in B$ there is at most one $x \in A$ such that~$f(x) =
    y$; equivalently if whenever $x \neq x'$ then $f(x) \neq f(x')$
    (see \olref[sfr][fun][kin]{sec})} }

\newglossaryentry{bijection} {
  name={bijection},  
  description={A function that is both {surjective} and
    {injective} (see \olref[sfr][fun][kin]{sec})} }

\newglossaryentry{enumeration} {
  name=enumeration,  
  description={A possibly infinite list of all
    {element}s of a set~$A$; formally a surjective function
    $f\colon \Nat \to A$ (see \olref[sfr][siz][enm]{sec})} }

\newglossaryentry{equinumerous} {
  name=equinumerous,
  description={$A$ and $B$ are equinumerous iff there is a total
    bijection from $A$ to $B$ (see \olref[sfr][siz][equ]{sec})} }

\newglossaryentry{finitely satisfiable} {
  name={finitely satisfiable},  
  description={$\Gamma$ is finitely satisfiable iff every finite
    $\Gamma_0 \subseteq \Gamma$ is satisfiable (see
    \olref[fol][com][com]{sec})} }

\newglossaryentry{mcs} {
  name=complete consistent set,
  description={A set of {sentence}s is complete and consistent iff it
    is consistent, and for every {sentence}~$!A$ either $!A$ or
    $\lnot !A$ is in the set (see \olref[fol][com][ccs]{sec})} }

\newglossaryentry{mos} {
  name=model,  
  description={A {structure} in which every {sentence}
    in~$\Gamma$ is true is a model of~$\Gamma$ (see
    \olref[fol][mat][exs]{sec})} }

\newglossaryentry{closed} {
  name=closed, 
  description={A set of {sentence}s~$\Gamma$ is closed iff, whenever
    $\Gamma \Entails !A$ then $!A \in \Gamma$. The set
    $\Setabs{!A}{\Gamma \Entails !A}$ is the closure of~$\Gamma$ (see
    \olref[fol][mat][int]{sec})} }

\newglossaryentry{derivability} {
  name=derivability,
  symbol = {\ensuremath{\Gamma \Proves !A}},  
  description={In the sequent calculus, $!A$ is {derivable} from~$\Gamma$ if there is a
    \gls{derivation} of a \gls{sequent} $\Gamma_0 \Sequent !A$ 
    where $\Gamma_0 \subseteq \Gamma$ is a finite sequence of sentences in~$\Gamma$ (see
    \olref[fol][seq][ptn]{sec}). In natural deduction, $!A$ is {derivable} from~$\Gamma$ if there is a
    {derivation} with end-{formula}~$!A$ and in which every
    assumption is either {discharged} or is in~$\Gamma$ (see
    \olref[fol][ntd][ptn]{sec})} }

\newglossaryentry{theorem} {
  name={theorem},
  symbol = {\ensuremath{\Proves !A}},
  description={In the sequent calculus, a {formula}~$!A$ is a theorem
    (of logic) if there is a {derivation} of the sequent $\Sequent !A$
    (see \olref[fol][seq][ptn]{sec}).  In natural deduction, a
    {formula}~$!A$ is a theorem if there is a {derivation} of $!A$
    with all assumptions \gls{discharged}
    (see \olref[fol][ntd][ptn]{sec}). We also say that $!A$ is a
    theorem of a theory~$\Gamma$ if $\Gamma \Proves !A$} }

\newglossaryentry{consistent} {
  name=consistent,
  description={In the sequent calculus, a set of sentences~$\Gamma$ is
    consistent iff there is no \gls{derivation} of a \gls{sequent}
    $\Gamma_0 \Sequent \quad$ with $\Gamma_0 \subseteq \Gamma$
    (see \olref[fol][seq][ptn]{sec}). In natural deduction, $\Gamma$
    is consistent iff $\Gamma \Proves/ \lfalse$
    (see \olref[fol][ntd][ptn]{sec}). If $\Gamma$ is not consistent,
    it is inconsistent.} }

\newglossaryentry{inconsistent} {
  name=inconsistent,
  description = {see \gls{consistent}}}

\newglossaryentry{derivation} {
  name=derivation,  
  description={In the sequent calculus, a tree of sequents in which every sequent is either an initial sequent or follows from the sequents immediately above it by a rule of
    inference (see \olref[fol][seq][rul]{sec}). In natural deduction, a tree of formulas in which every formula is either an
    assumption or follows from the formulas immediately above it by a rule of
    inference (see \olref[fol][ntd][rul]{sec})} }

\newglossaryentry{eigenvariable}{
  name = eigenvariable,  
  description = {In the sequent calculus, a special constant symbol in a premise of a
    $\LeftR{\lexists}$ or $\RightR{\lforall}$ inference which may not
    appear in the conclusion (see
    \olref[fol][seq][rul]{sec}). In natural deduction, a special constant symbol in a premise of a
    $\Elim{\lexists}$ or $\Intro{\lforall}$ inference which may not
    appear in the conclusion or any \gls{undischarged} assumption (see
    \olref[fol][ntd][rul]{sec})}}

\newglossaryentry{discharged}{
  name = {discharged},
  description = {An \gls{assumption} in a derivation may be discharged
    by an inference rule below it (the rule and the assumption are
    then assigned a matching label, e.g., $[!A]^2$). If it is not
    discharged, it is called undischarged (see \olref[fol][ntd][rul]{sec})}}

\newglossaryentry{undischarged}{
  name = {undischarged},
  description = {see \gls{discharged}}}

\newglossaryentry{assumption}{
  name = {assumption},
  description = {A formula that stands topmost in a derivation, also
    called an initial formula. It may be \gls{discharged} or
    undischarged (see \olref[fol][ntd][rul]{sec})}}

\newglossaryentry{soundness} {
  name=soundness, 
  description={Property of a !!{derivation} system: it is sound if whenever
    $\Gamma \Proves !A$ then $\Gamma \Entails !A$ (see
    \olref[fol][seq][sou]{sec} and \olref[fol][ntd][sou]{sec})} }

\newglossaryentry{sequent}{
  name=sequent,
  description = {An expression of the form $\Gamma \Sequent \Delta$ where $\Gamma$ and $\Delta$ are finite sequences of sentences (see \olref[fol][seq][rul]{sec})} }


\newglossaryentry{sentence} {
  name=sentence,  
  description={A {formula} with no \gls{free} variable. (see
    \olref[fol][syn][fvs]{sec})} }

\newglossaryentry{variable assignment} {
  name=variable assignment,  
  description={A \gls{function} which maps each {variable} to an element
    of~$\Domain M$ (see \olref[fol][syn][sat]{sec})} }

\newglossaryentry{$x$-variant} {
  name=$x$-variant,
  sort={x-variant},
  description={Two \glspl{variable assignment} are $x$-variants, $s \sim_x
    s'$, if they differ at most in what they assign to~$x$ (see
    \olref[fol][syn][sat]{sec})} }

\newglossaryentry{valid} {
  name=valid,
  symbol = {\ensuremath{\Entails !A}},
  description={A sentence $!A$ is \emph{valid}  iff
    $\Sat{M}{!A}$ for every \gls{structure}~$\Struct M$ (see
    \olref[fol][syn][sem]{sec})} }

\newglossaryentry{entailment} {
  name=entailment,
  symbol = {\ensuremath{\Gamma \Entails !A}},
  description={A set of sentences~$\Gamma$ entails a
    sentence~$!A$ iff for every
    \gls{structure}~$\Struct M$ with $\Sat{M}{\Gamma}$, $\Sat{M}{!A}$
    (see \olref[fol][syn][sem]{sec})} }

\newglossaryentry{satisfiable} {
  name=satisfiable,
  description={A set of sentences~$\Gamma$ is satisfiable if
    $\Sat{M}{\Gamma}$ for some \gls{structure}~$\Struct M$, otherwise it
    is unsatisfiable (see \olref[fol][syn][sem]{sec})} }

\newglossaryentry{cs} {
    name=covered,    
    description={A \gls{structure} in which every element of the
      domain is the {value} of some closed term (see
      \olref[fol][syn][str]{sec})} }

\newglossaryentry{subformula} {
  name={subformula},  
  description={Part of a formula which is itself a formula (see
    \olref[fol][syn][sbf]{sec})} }

\newglossaryentry{free for} {
  name=free for,
  description={A term~$t$ is {free for} $x$ in $!A$ if none
    of the \gls{free} occurrences of~$x$ in $!A$ occur in the scope of a
    quantifier that binds a variable in~$t$ (see
    \olref[fol][syn][sub]{sec})} }

\newglossaryentry{church-turing thesis} {
  name=Church-Turing Thesis,
  description={states that anything computable via an effective
    procedure is Turing computable (see \olref[tur][mac][ctt]{sec})} }

\newglossaryentry{completeness} {
  name=completeness, 
  description={Property of a !!{derivation} system; it is complete if, whenever
    $\Gamma$ entails $!A$, then there is also a {derivation} that
    establishes $\Gamma \Proves !A$; equivalently, iff
    every \gls{consistent} set of sentences is \gls{satisfiable}
    (see \olref[fol][com][int]{sec})} }

\newglossaryentry{formula} {
  name={formula},  
  description={Expressions of a first-order language~$\Lang L$ which
    express relations or properties, or are true or false (see
    \olref[fol][syn][frm]{sec})} }

\newglossaryentry{structure} {
  name={structure},
  symbol = {\ensuremath{\Struct{M}}},
  description={An interpretation of a first-order language, consisting
    of a \gls{sdomain} and assignments of the constant,
    predicate and function symbols of the language (see
    \olref[fol][syn][str]{sec})} }

\newglossaryentry{sdomain} {
  name = {domain (of a structure)},
  symbol = {\ensuremath{\Domain{M}}},
  description = {Non-empty set from from which a \gls{structure} takes
    assignments and values of {variables} (see
    \olref[fol][syn][str]{sec})} }
 
\newglossaryentry{halting problem} {
  name=halting problem,
  description={The problem of determining (for any $e$, $n$) whether
    the Turing machine~$M_e$ halts for an input of~$n$ strokes (see
    \olref[tur][und][hal]{sec})} }

\newglossaryentry{bound} {
  name=bound,
  description={Occurrence of a variable
    within the scope of a quantifier that uses the same
    variable (see \olref[fol][syn][fvs]{sec})}
}

\newglossaryentry{free} {
  name=free,
    description={An occurrence of a variable that is not \gls{bound}
    (see \olref[fol][syn][fvs]{sec})}
}

\newglossaryentry{compactness theorem} {
  name = {compactness theorem},  
  description = {States that every \gls{finitely satisfiable} set of
    sentences is satisfiable (see
    \olref[fol][com][com]{sec})}}

\newglossaryentry{ls}{
  name = {L\"owenheim-Skolem Theorem},  
  description = {States that every satisfiable set of sentences has a
    countable model (see \olref[fol][com][dls]{sec})}}

\newglossaryentry{Church-Turing Theorem}{
  name = {Church-Turing Theorem},
  description = {States that there is no Turing machine which decides if a
    given sentence of first-order logic is \gls{valid} or not (see
    \olref[tur][und][uns]{sec})}}

\newglossaryentry{decision problem}{
  name = {decision problem}, 
  description = {Problem of deciding if a given sentence of
    first-order logic is \gls{valid} or not (see \gls{Church-Turing
    Theorem})}}

\newglossaryentry{completeness-theorem}{
  name = {completeness theorem},  
  description = {States that first-order logic is complete: every
    \gls{consistent} set of sentences is \gls{satisfiable}}}


