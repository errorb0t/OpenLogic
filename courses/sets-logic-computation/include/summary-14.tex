A \emph{Turing machine} is a kind of idealized computation
mechanism. It consists of a one-way infinite \emph{tape}, divided into
squares, each of which can contain a \emph{symbol} from a
pre-determined alphabet. The machine operates by moving a
\emph{read-write head} along the tape. It may also be in one of a
pre-determined number of \emph{states}. The actions of the read-write
head are determined by a set of instructions; each instruction is
conditional on the machine being in a certain state and reading a
certain symbol, and specifies which symbol the machine will write onto
the current square, whether it will move the read-write head one
square left, right, or stay put, and which state it will switch to.
If the tape contains a certain \emph{input}, represented as a sequence
of symbols on the tape, and the machine is put into the designated
start state with the read-write head reading the leftmost square of
the input, the instruction set will step-wise determine a sequence of
\emph{configurations} of the machine: content of tape, position of
read-write head, and state of the machine. Should the machine
encounter a configuration in which the instruction set does not
contain an instruction for the current symbol read/state combination,
the machine \emph{halts}, and the content of the tape is the
\emph{output}.

Numbers can very easily be represented as sequences of strokes on the
Tape of a Turing machine. We say a function $\Nat \to \Nat$ is
\emph{Turing computable} if there is a Turing machine which, whenever
it is started on the unary representation of~$n$ as input, eventually
halts with its tape containing the unary representation of~$f(n)$ as
output. Many familiar arithmetical functions are easily (or
not-so-easily) shown to be Turing computable. Many other models of
computation other than Turing machines have been proposed; and it has
always turned out that the arithmetical functions computable there are
also Turing computable. This is seen as support for the
\emph{Church-Turing Thesis}, that every arithmetical function that can
effectively be computed is Turing computable.
