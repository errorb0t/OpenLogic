Eine \emph{Relation}~$R$ auf einer Menge~$A$ ist eine Art, Elemente
aus~$A$ zueinander in Beziehung zu setzen. Wir schreiben $Rxy$, wenn die Beziehung zwischen $x$
und~$y$ \emph{besteht}. Formal können wir $R$ betrachten als die Menge der Paare
$\tuple{x,y} \in A^2$ sodass~$Rxy$ GILT. Kleiner-als, größer-als,
gleich, restlos teilend, gleich-lang-wie, eine-Teilmenge-von und gleich-groß-wie
sind allesamt wichtige Beispiele für Relationen (zu
Mengen von Zahlen, Zeichenketten oder Mengen). \emph{Graphen} sind eine allgemeine Methode
um Beziehungen visuell darzustellen. Ein Graph kann aber auch als eine
binäre Relation (die \emph{Kanten}-Relation) zusammen mit der
zugrundeliegenden Menge von \emph{Punkten} verstanden werden.

Einige Relationen haben bestimmte Eigenschaften, die sie besonders
interessant oder nützlich machen. Eine Relation~$R$ ist \emph{reflexiv}, wenn
alles unter $R$ zu sich selbst in Beziehung steht; \emph{symmetrisch}, wenn für  $Rxy$
auch $Ryx$ für beliebige $x$ und $y$ gilt; und \emph{transitiv}, wenn $Rxy$
und $Ryz$ garantieren, dass~$Rxz$. Relationen, die alle drei dieser
Eigenschaften besitzen, heißen \emph{Äquivalenzrelationen}. Eine Relation ist
\emph{antisymmetrisch}, wenn $Rxy$ und $Ryx$
garantieren, dass~$x=y$. \emph{Halbordnungen} sind solche Relationen, die
reflexiv, antisymmetrisch und transitiv sind. Eine \emph{lineare Ordnung} ist
jede Halbordnung, für die gilt, dass für jedes $x$ und~$y$ entweder
$Rxy$ oder $x=y$ oder $Ryx$. (Im Allgemeinen heißt eine Beziehung mit dieser Eigenschaft
\emph{verbunden}).

Da Relationen Mengen (von Paaren) sind, kann man mit ihnen wie mit Mengen operieren
(z. B. können wir die Vereinigung und die Schnittmenge von Relationen bilden). Wir können
sie auch miteinander verketten (\emph{relatives Produkt}~$R \mid S$). Wenn wir
das relative Produkt von~$R$ mit sich selbst beliebig oft bilden, erhalten wir die \emph{transitive Hülle}~$R^+$ von~$R$.


Übersetzt mit www.DeepL.com/Translator (kostenlose Version)