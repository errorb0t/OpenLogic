The size of a set~$A$ can be measured by a natural number if the set
is finite, and sizes can be compared by comparing these numbers. If sets
are infinite, things are more complicated. The first level of
infinity is that of \emph{countably infinite} sets. A set~$A$ is
countable if its elements can be arranged in an \emph{enumeration}, a
one-way infinite list, i.e., when there is a
surjective function~$f\colon \PosInt \to A$. It is countably infinite
if it is countable but not finite. Cantor's \emph{zig-zag method}
shows that the sets of pairs of elements of countably infinite sets is
also countable; and this can be used to show that even the set of
rational numbers~$\Rat$ is countable.

There are, however, infinite sets that are not countable: these sets
are called \emph{uncountable}. There are two ways of showing that a
set is uncountable: directly, using a \emph{diagonal argument}, or by
\emph{reduction}. To give a diagonal argument, we assume that the
set~$A$ in question is countable, and use a hypothetical enumeration
to define an element of~$A$ which, by the very way we define it, is
guaranteed to be different from every element in the enumeration. So
the enumeration can't be an enumeration of all of~$A$ after all, and
we've shown that no enumeration of~$A$ can exist. A reduction shows
that $A$ is uncountable by associating every element of~$A$ with an
element of some known uncountable set~$B$ in a surjective way. If
this is possible, than a hypothetical enumeration of $A$ would yield
an enumeration of~$B$. Since $B$ is uncountable, no enumeration of~$A$
can exist.

In general, infinite sets can be compared sizewise: $A$ and $B$ are
the same size, or \emph{equinumerous}, if there is a bijection between
them. We can also define that $A$ is no larger than~$B$
($\cardle{A}{B}$) if there is an injective function from $A$ to~$B$.
By the Schr\"oder-Bernstein Theorem, this in fact provides a sizewise
order of infinite sets. Finally, \emph{Cantor's theorem} says that for
any $A$, $\cardless{A}{\Pow{A}}$. This is a generalization of our
result that $\Pow{\PosInt}$ is uncountable, and shows that there are
not just two, but infinitely many levels of infinity.
