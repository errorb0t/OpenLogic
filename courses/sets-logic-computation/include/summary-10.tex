\emph{Proof systems} provide purely syntactic methods for
characterizing consequence and compatibility between sentences.
The \emph{sequent calculus} is one such proof
system. A \emph{derivation} in it consists of a tree of sequents (a
sequent $\Gamma \Sequent \Delta$ consists of two sequences of formulas
separated by~$\Sequent$). The topmost sequents in a derivation
are \emph{initial sequents} of the form $!A \Sequent !A$.  All other
sequents, for the derivation to be correct, must be correctly
justified by one of a number of \emph{inference rules}. These come in
pairs; a rule for operating on the left and on the right side of a
sequent for each connective and quantifier. For instance, if a sequent
$\Gamma \Sequent \Delta, !A \lif !B$ is justified by the
$\RightR{\lif}$ rule, the preceding sequent (the \emph{premise}) must
be $!A, \Gamma \Sequent \Delta, !B$. Some rules also allow the order
or number of sentences in a sequent to be manipulated, e.g.,
the \RightR{\Exchange} rule allows two formulas on the right side of a
sequent to be switched.

If there is a derivation of the sequent $\quad \Sequent !A$, we say
$!A$ is a \emph{theorem} and write~$\Proves !A$. If there is a
derivation of $\Gamma_0 \Sequent !A$ where every $!B$ in $\Gamma_0$ is
in~$\Gamma$, we say $!A$ is \emph{derivable from}~$\Gamma$ and write
$\Gamma \Proves !A$. If there is a derivation of
$\Gamma_0 \Sequent \quad$ where every $!B$ in $\Gamma_0$ is
in~$\Gamma$, we say $\Gamma$ is \emph{inconsistent}, otherwise
\emph{consistent}. These notions are interrelated, e.g., $\Gamma
\Proves !A$ iff $\Gamma \cup \{\lnot !A\}$ is inconsistent. They are
also related to the corresponding semantic notions, e.g., if $\Gamma
\Proves !A$ then $\Gamma \Entails !A$. This property of proof
systems---what can be derived from $\Gamma$ is guaranteed to be
entailed by~$\Gamma$---is called \emph{soundness}. The \emph{soundness
theorem} is proved by induction on the length of derivations, showing
that each individual inference preserves validity of the conclusion
sequent provided the premise sequents are valid.
