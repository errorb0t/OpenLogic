A \emph{set} is a collection of objects, the elements of the set. We
write $x \in A$ if $x$~is an element of~$A$. Sets are
\emph{extensional}---they are completely determined by their elements.
Sets are specified by \emph{listing} the elements explicitly or by
giving a property the elements share
(\emph{abstraction}). Extensionality means that the order or way of
listing or specifying the elements of a set doesn't matter. To prove
that $A$ and~$B$ are the same set ($A = B$) one has to prove that
every element of~$X$ is an element of~$Y$ and vice versa.

Important sets include the natural ($\Nat$), integer ($\Int$),
rational ($\Rat$), and real ($\Real$) numbers, but also \emph{strings}
($X^*$) and infinite \emph{sequences} ($X^\omega$) of objects. $A$~is
a \emph{subset} of $B$, $A \subseteq B$, if every element of $A$ is
also one of~$B$. The collection of all subsets of a set~$B$ is itself
a set, the \emph{power set} $\Pow{B}$ of~$B$. We can form the
\emph{union} $A \cup B$ and \emph{intersection} $A \cap B$ of sets. An
\emph{ordered pair} $\tuple{x, y}$ consists of two objects $x$
and~$y$, but in that specific order. The pairs $\tuple{x, y}$ and
$\tuple{y, x}$ are different pairs (unless $x = y$). The set of all
pairs $\tuple{x, y}$ where $x \in A$ and $y \in B$ is called the
\emph{Cartesian product}~$A \times B$ of $A$ and $B$. We write $A^2$
for $A \times A$; so for instance $\Nat^2$ is the set of pairs of
natural numbers.
