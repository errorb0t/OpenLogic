Eine \emph{Menge} ist eine Sammlung von Objekten, den Elementen der Menge. Wir
schreiben $x \in A$, falls $x$~ein Element aus~$A$ ist. Mengen sind
\emph{extensional}---sie sind vollständig durch ihre Elemente bestimmt.
Mengen werden gebildet durch explizite Auflistung der Elemente oder durch
Angabe einer Eigenschaft, die die Elemente gemeinsam haben
(\emph{Abstraktion}). Extensionalität bedeutet, dass die Reihenfolge oder die Art der
Auflistung oder Angabe der Elemente einer Menge keine Rolle spielt. Um zu beweisen,
dass $A$ und~$B$ die gleiche Menge sind ($A = B$), muss man beweisen, dass
jedes Element aus~$X$ ein Element aus~$Y$ ist und umgekehrt.

Wichtige Mengen sind die natürlichen ($\Nat$), die ganzen ($\Int$),
die rationalen ($\Rat$) und die reellen ($\Real$) Zahlen, aber auch \emph{Zeichenfolgen}
($X^*$) und unendliche \emph{Folgen} ($X^\omega$) von Objekten. $A$~ist
eine \emph{Teilmenge} von $B$, $A \subseteq B$, wenn jedes Element von $A$
auch eines von~$B$ ist. Die Sammlung aller Teilmengen einer Menge~$B$ ist selbst
eine Menge, die \emph{Potenzmenge} $\Pow{B}$ von~$B$. Wir können
\emph{Vereinigung} $A \cap B$ und \emph{Durchschnitt} $A \cap B$ von Mengen bilden. Ein
\emph{geordnetes Paar} $\tuple{x, y}$ besteht aus zwei Objekten $x$
und~$y$, aber in dieser bestimmten Reihenfolge. Die Paare $\tuple{x, y}$ und
$\tuple{y, x}$ sind verschiedene Paare (es sei denn $x = y$). Die Menge aller
Paare $\tuple{x, y}$, bei denen $x \in A$ und $y \in B$ ist, heißt das
\emph{Kartesische Produkt}~$A \times B$ von $A$ und $B$. Wir schreiben $A^2$
für $A \times A$; so ist zum Beispiel $\Nat^2$ die Menge der Paare von
natürlichen Zahlen.
