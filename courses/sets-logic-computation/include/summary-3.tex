Eine \emph{Funktion}~$f\colon A \to B$ bildet jedes Element des
\emph{Definitionsbereichs}~$A$ auf genau ein Element des \emph{Wertebereichs}~$B$ ab. Für
$x \in A$, nennen wir das $y$, das entsteht, wenn $x$ von $f$ abgebildet wird, den
\emph{Wert}~$f(x)$ von $f$~für ein \emph{Argument}~$x$. Wenn $A$ eine Menge von
Paaren ist, kann man sich vorstellen, dass die Funktion~$f$ zwei Argumente annimmt. Das
\emph{Bild}~$\ran{f}$ von~$f$ ist die Teilmenge von~$B$, die aus
allen Werten von~$f$ besteht.

Der Wert~$f(x)$ ist insofern eindeutig, als $f$ jedes $x$ nur auf ein $f(x)$ abbildet,
niemals auf mehr als eines. Wenn $\ran{f} = B$ ist, dann heißt $f$ \emph{surjektiv}.
Wenn $f(x)$ auch in dem Sinne eindeutig ist, dass keine zwei verschiedenen Argumente
auf denselben Wert abgebildet werden, nennt man $f$ \emph{injektiv}. Funktionen,
die sowohl injektiv als auch surjektiv sind, werden \emph{bijektiv} genannt.

Bijektive Funktionen haben eine eindeutige \emph{Inverse}~$f^{-1}$.
Funktionen können auch miteinander verkettet werden: die Funktion $(g \circ f)$ ist
die \emph{Komposition} von $f$ mit $g$. Kompositionen von injektiven
Funktionen sind injektiv, und Kompositionen von surjektiven Funktionen sind surjektiv. 
$(f^{-1} \circ f)$ ist die Identitätsfunktion.

Wenn wir die Bedingung lockern, dass $f$ für jedes $x \in A$ einen Wert haben muss,
erhalten wir den Begriff der \emph{Teilfunktionen}. Wenn $f\colon
A \pto B$ partiell ist, nennen wir $f(x)$ \emph{definiert},
$f(x) \fdefined$, wenn $f$ einen Wert für das Argument~$x$ hat, und sonst \emph{undefiniert}, $f(x) \fundefined$.
Jede (partielle)
Funktion~$f$ ist mit dem \emph{Graphen}~$R_f$ von $f$ verbunden, der
Beziehung, die gilt gdw $f(x) = y$.
