Sets of sentences in a sense describe the structures in which they are
jointly true; these structures are their \emph{models}. Conversely,
if we start with a structure or set of structures, we might be
interested in the set of sentences they are models of, this is the
\emph{theory} of the structure or set of structures. Any such set of
sentences has the property that every sentence entailed by them is
already in the set; they are \emph{closed}. More generally, we call a
set $\Gamma$ a theory if it is closed under entailment, and say
$\Gamma$ is \emph{axiomatized} by~$\Delta$ is $\Gamma$ consists of all
sentences entailed by~$\Delta$.

Mathematics yields many examples of theories, e.g., the theories of
linear orders, of groups, or theories of arithmetic, e.g., the theory
axiomatized by Peano's axioms. But there are many examples of
important theories in other disciplines as well, e.g., relational
databases may be thought of as theories, and metaphysics concerns
itself with theories of parthood which can be axiomatized.

One significant question when setting up a theory for study is whether
its language is expressive enough to allow us to formulate everything
we want the theory to talk about, and another is whether it is strong
enough to prove what we want it to prove. To \emph{express} a
relation we need a formula with the requisite number of free
variables. In \emph{set theory}, we only have $\in$ as a relation
symbol, but it allows us to express $x \subseteq y$ using
$\lforall[u][(u \in x \lif u \in y)]$. \emph{Zermelo-Fraenkel set
  theory}~$\Log{ZFC}$, in fact, is strong enough to both express
(almost) every mathematical claim and to (almost) prove every
mathematical theorem using a handful of axioms and a chain of
increasingly complicated definitions such as that of~$\subseteq$.
